% !TEX root = exercises.tex

\exercisetitle{Exercise Sheet 0}

\begin{questions}
\question Write the following complex numbers in polar form $z=r \left( \cos ( \theta)+ i \sin (\theta) \right)$ (or equivalently, $z=r \exp \left(i \theta \right)$):
\begin{parts}
\part $1+i$
\part $-1+i$
\part $1+i \sqrt{3}$
\part $\dfrac{(1+i)^7}{(1+i\sqrt{3})^2}$
\end{parts}

\begin{answer}
\begin{enumerate}
\item[(a)] $1+i=\sqrt{2} \left( \cos ( \frac{\pi}{4} ) + i \sin ( \frac{\pi}{4} ) \right)$
\item[(b)] $-1+i=\sqrt{2} \left( \cos ( \frac{3\pi}{4} ) + i \sin ( \frac{3\pi}{4} ) \right)$
\item[(c)] $1+i\sqrt{3}=2 \left( \cos ( \frac{\pi}{3} ) + i \sin ( \frac{\pi}{3} ) \right)$
\item[(d)] While we could expand the numerator and denominator and then simplify, it is much easier to use De Moivre's Theorem:
\[
r \left( \cos ( \theta) + i \sin (\theta ) \right)^n = r^n \left( \cos ( n\theta) + i \sin ( n \theta ) \right) \quad (n \in \mathbb{Z}).
\]
\begin{align*}
(1+i)^7 & = \polar{2^{7/2}}{7 \pi / 4},\\ 
(1+ i \sqrt{3})^{-2} & = \polar{2^{-2}}{-2\pi/3}.
\end{align*}
Then we get
\begin{align*}
\frac{(1+i)^7}{(1+i\sqrt{3})^2} & = (1+i)^7(1+i \sqrt{3})^{-2} \\
& = \polar{2^{3/2}}{13\pi/12}. 
\end{align*}
This is fine, however, if we wish to use the principal value of the argument, we must subtract an appropriate integer multiple of $2\pi$ from $13\pi/12$ (to ensure that the argument lies in $(-\pi,\pi]$).  This is given by  $13\pi /12-2\pi = -11\pi/12$, thus
\[
\frac{(1+i)^7}{(1+i\sqrt{3})^2} = \polar{2^{3/2}}{-11\pi /12}.
\]
An equally valid way of answering this is using the exponential polar form:
\[
(1+i)^7 = \expform{2^{7/2}}{7 \pi / 4}\qquad(1+ i \sqrt{3})^{-2} = \expform{2^{-2}}{-2\pi/3}.
\]
The properties $\exp(z_1)\exp(z_2)=\exp(z_1+z_2)$ and $\exp(z+2\pi i)=\exp(z)$  give:
\[
\frac{(1+i)^7}{(1+i\sqrt{3})^2} = \expform{2^{7/2}}{7 \pi/4} \expform{2^{-2}}{-2\pi/3}  =  \expform{2^{3/2}}{-11\pi/12}.
\]
\end{enumerate}
\end{answer}

\question 
\begin{parts}
\part For $z=a+ib$, express $z^{-1}$ (i.e, $\dfrac{1}{z}$) in Cartesian form
\part Write down the modulus and Principal Argument of $z^{-1}$ in terms of $\abs{z}$ and $\Arg (z)$.
\end{parts}

\begin{answer}
\begin{enumerate}
\item[(a)] We use the fact that $\displaystyle \frac{1}{z} = \frac{1}{z} \cdot \frac{\conj{z}}{\conj{z}} = \frac{\conj{z}}{\abs{z}^2}$ to invert any nonzero complex number:
\[
\frac{1}{a+ib} = \frac{1}{a+ib} \cdot \frac{a-ib}{a-ib} = \frac{a-ib}{a^2+b^2} = \frac{a}{a^2+b^2} - i\frac{b}{a^2+b^2}.
\]
\item[(b)] Writing $z=\polar{r}{\theta}$, De Moivre's Theorem gives
\[
z^{-1} = \left(\polar{r}{\theta}\right)^{-1} = \polar{\frac{1}{r}}{-\theta}.
\]
Hence $\abs{z^{-1}} = 1/\abs{z}$ and $\Arg (z^{-1})=-\Arg (z)$.
\end{enumerate}
\end{answer}

\question Express the following complex numbers in Cartesian form (that is to say, as $a+ib$ for $a,b \in \R$)
\[
\frac{i-1}{1-i}\qquad \frac{1}{1+i}\qquad \frac{3+4i}{1-2i}.
\]
\begin{answer}
In each case, we make the denominator of the quotient $\frac{z_1}{z_2}$ real by multiplying by $\conj{z_2}/ \conj{z_2}$, which gives
\[
\frac{z_1}{z_2} = \frac{z_1\conj{z_2}}{\abs{z_2}^2}.
\]
Thus for example
\begin{align*}
\frac{i-1}{1-i} & = \frac{i-1}{1-i} \cdot \frac{1+i}{1+i} \\
& = \frac{1}{2} \left[ i(1+i)-1(1+i) \right] \\
& = \frac{1}{2} \left[-2 \right] \\
& = -1\ (+i0).
\end{align*}
Similarly
\begin{align*}
\frac{1}{1+i} &=  \frac{1}{2} - \frac{i}{2} \\
\frac{3+4i}{1-2i} &= -1+2i.
\end{align*}
\end{answer}

\question Find the principal argument of the four points $\pm 1\pm i\sqrt{3}$.
\begin{answer}
\begin{align*}
\Arg (1+i\sqrt{3}) &= \frac{\pi}{3} \\
\Arg (-1+i \sqrt{3} ) &= \frac{2\pi}{3} \\
\Arg (-1-i\sqrt{3} ) &= - \frac{2\pi}{3} \\
\Arg (1-i\sqrt{3} ) &= - \frac{\pi}{3}. 
\end{align*}
\end{answer}




\question Calculate
\[
\Arg \left. \left( \frac{1}{2} + \frac{1}{z^2} \right) \right|_{z=1+i}.
\]
(The notation
$
\left. f(z) \right|_{z=w}
$
means $f(z)$ evaluated at $z=w$, or in other words, $f(w)$.)
\begin{answer}
We have
\[
\frac{1}{2} + \frac{1}{(1+i)^2} = \frac{1}{2}-\frac{i}{2},
\]
with principal argument $-\frac{\pi}{4}$.
\end{answer}
\question Show that
\[
2 \left( \frac{z}{z+i} \right) \left. \frac{(z+i-z)}{(z+i)^2} \right|_{z=i} = \frac{-i}{4}.
\]
\begin{answer}
\begin{align*}
2 \left( \frac{z}{z+i} \right) \left. \frac{(z+i-z)}{(z+i)^2} \right|_{z=i} & = 2 \left( \frac{i}{2i} \right) \left( \frac{i+i-i}{(2i)^2} \right) \\
& = \frac{2i}{2i} \frac{i}{(-4)} = -\frac{i}{4}.
\end{align*}
\end{answer}


\question Sketch the following regions of $\C$:
\begin{parts}
\part $1 < |z| < 2$ (This notation is shorthand for the set $\set{ z \in \C: 1 < \abs{z} <2}$.)
\part $1<|z+2|<2$
\part $1 < \Im (z-i) < 2$
\end{parts}
\begin{answer}

\begin{enumerate}
\item[(a),(b)] Note that for $\alpha \in \C$ and $r>0$, the set $\set{ z \in \C: \abs{z-\alpha} = r}$ is precisely the circle of radius $r$ centred at $\alpha$.  The sets $0<\abs{z-\alpha}<r$ and $\abs{z-\alpha}>r$ respectively are the regions inside and outside the circle.

Hence $1<z<2$ is the region outside the circle of radius 1 centred at 0, and inside the circle of radius 2 centred at 0.  The set $1<\abs{z+2}<2$ is the same but with circles centred at $-2$.  A set of this type is called an \emph{annulus}.
\begin{center}
\includegraphics[scale=0.75]{annulus}\quad \includegraphics[scale=0.75]{annulus2}
\end{center}

\item[(c)] Write $z=x+iy$, then $z-i = x+i(y-1)$ and so
\begin{align*}
\set{z\in \C: 1 < \Im(z-i) <2 } &= \set{ x+iy \in \C : 1<y-1 <2 } \\
&= \set{x+iy \in \C: 2<y<3},
\end{align*}
which is the infinite horizontal band bounded by the lines $y=2$ and $y=3$.
\begin{center}
\oldincludegraphics[scale=0.75]{band}
\end{center}
\end{enumerate}
\end{answer}
\question 
\begin{parts}
\part Prove that for two nonzero complex numbers $z_1$ and $z_2$ we have
\[
\abs{z_1z_2} = \abs{z_1} \cdot \abs{z_2} \quad\text{ and }\quad \arg(z_1z_2)=\arg(z_1)\arg(z_2)
\]
(hint: write $z_1$ and $z_2$ in polar form).  Is it always true that $\Arg(z_1z_2)=\Arg(z_2)+\Arg(z_2)$?
\begin{answer}
Writing
\[
z_1=\polar{r_1}{\theta_1}\quad\text{and}\quad z_2=\polar{r_2}{\theta_2}
\]
we see that
\begin{align*}
z_1z_2&=r_1r_2 \left( \cos(\theta_1)\cos(\theta_2)-\sin(\theta_1)\sin(\theta_2) +i \left[ \cos(\theta_1)\sin(\theta_2)+\cos(\theta_2)\sin(\theta_1) \right] \right) \\
& = r_1r_2 \left( \cos(\theta_1+\theta_2)+i \sin(\theta_1+\theta_2) \right).
\end{align*}
Thus
\[
\abs{z_1z_2}=r_1r_2=\abs{z_1}\cdot \abs{z_2}\quad\text{and}\quad \arg(z_1z_2)=\theta_1+\theta_2=\arg(z_1)+\arg(z_2).
\]
It is not always true that $\Arg(z_1z_2)=\Arg(z_1)+\Arg(z_2)$.  Indeed, if $z_1=z_2=-1$ we have $\Arg (-1)=\pi$ but
\[
\Arg ((-1)(-1)) = \Arg (1) = 0.
\]
\end{answer}
\part Show that $\exp(z_1)\exp(z_2) = \exp(z_1+z_2)$.
\begin{answer}
With $z_1=x_1+iy_1$ and $z_2=x_2+iy_2$ a similar calculation to part (a) shows that
\begin{align*}
\exp(z_1)\exp(z_2) &= \polar{e^{x_1}}{y_1} \polar{e^{x_2}}{y_2} \\
 & = \polar{e^{x_1}e^{x_2}}{y_1+y_2} \\
 & = \polar{e^{x_1+x_2}}{y_1+y_2} = \exp(z_1+z_2).
\end{align*}
\end{answer}
\end{parts}
\question Write down the $3^{rd}$ roots of $-8$ in Cartesian form.
\begin{answer}
Writing $-8=\polar{8}{\pi}$, the three cubic roots are of the form $\polar{\sqrt[3]{8}}{(\pi+2k\pi)/3}$ for $k=0,1,2$.  In polar form, these are
\[
\polar{2}{\pi/3}, \quad \polar{2}{\pi} \quad\text{and}\quad\polar{2}{5\pi/3},
\]
or in Cartesian form $1+\sqrt{3},\ -2$ and $1-\sqrt{3}$ respectively.

\end{answer}

\question Find the values of $z$ for which $z^2+4iz-1=0$.  Which of these values lies inside the circle $C=\set{z \in \C: \abs{z}=1}$.
\begin{answer}
This is simply the usual quadratic formula: the roots of this polynomial are given by
\[
\frac{-4i\pm \sqrt{(4i)^2-4(1)(-1)}}{2} = \frac{-4i\pm i \sqrt{12}}{2} = i (-2 \pm  \sqrt{3}).
\]
To see which of these lies inside the circle $\abs{z}=1$, we examine the modulus and see that $\abs{i(-2 + \sqrt{3})} \approx 0.071<1$ and $\abs{i(-2-\sqrt{3})} \approx 1.93>1$. Thus only $i(-2+\sqrt{3})$ lies inside this circle.
\end{answer}


\question Show that $\Re (z) \leq \abs{ \Re (z)} \leq \abs{z}$ and $\abs{\Re (z)}+\abs{\Im (z)} \leq \sqrt{2} \abs{z}$.
\begin{answer}
The fact that $\Re (z) \leq \abs{ \Re (z)}$ is trivial.  Moreover,  writing $z=x+iy$ we see that
\[
\abs{\Re(z)} = \sqrt{x^2} \leq \sqrt{x^2+y^2}
\]
since $x^2 \leq x^2+y^2$ (and the square root function is increasing).

For the second inequality, note that with $z=x+iy$,
\begin{align*}
\left( \abs{\Re (z)}+\abs{\Im (z)} \right)^2 & = \left( \abs{x}+\abs{y} \right)^2 \\
& \leq \left( \abs{x}+\abs{y} \right)^2  + \overbrace{\left( \abs{x}-\abs{y} \right)^2}^{\geq 0}  \\
& = x^2+y^2+2\abs{xy} + x^2 + y^2 - 2\abs{xy} \\
& = 2(x^2+y^2) = 2 \abs{z}^2.
\end{align*}
Taking square roots of both sides yields the required inequality.
\end{answer}
\end{questions}
