% !TEX root = exercises.tex

\exercisetitle{Exercise Sheet 5}

\begin{questions}
\question Locate the poles of each of the following functions, and calculate the residues at these poles:
\begin{parts}
\part $f(z) = \dfrac{1}{z(i-z)^3}$
\part $f(z) = \dfrac{z^2}{(z^2+1)^2}$
\part $f(z) = \dfrac{\Log(z)}{(4z-i)^2}$
\part $f(z)=\dfrac{1}{\exp(z)-1}$.
\end{parts}
\begin{answer}
\begin{parts}
\part This time $f$ has a pole of order $1$ at $0$ and a pole of order $3$ at $i$.  With $g_1(z) = \frac{1}{(i-z)^3}$ and $h_1(z)=z$, the $g/h$ rule gives
\[
\Res (f;0) = \frac{g(0)}{h'(0)} = \frac{1}{(1) \cdot (i)^3}  = i.
\]
For the pole of order $3$ at $i$ we first rewrite
\[
f(z) = \frac{1}{z(-(z-i))^3} = - \frac{1}{z(z-i)^3}.
\]
Now, letting $g_2(z) = -z^{-1}$ we can use the formula
\[
\Res (f;i) = \frac{g_2''(i)}{2!}.
\]
We have $g_2'(z) = z^{-2}$ and $g_2''(z) = -2z^{-3}$, so that $g_2''(i)=-2/(i)^3 = -2i$ and hence
\[
\Res (f;i) = \frac{-2i}{2} = -i.
\]
\part Writing
\[
f(z) = \frac{z^2}{(z+i)^2(z-i)^2}
\]
we see that $f$ has poles of order $2$ at $z=\pm i$.  If $g_1 (z)=  \dfrac{z^2}{(z+i)^2}$ then
\[
g_1'(z) = \frac{(z+i)^2(2z)-2(z+i)z^2}{(z+i)^4},\quad g_1'(i) = \frac{(2i)^3-2(2i)i^2}{(2i)^4} = - \frac{i}{4}
\]
hence
\[
\Res (f; i) = \frac{g_1'(i)}{1!} = - \frac{i}{4}.
\]
Similarly, with $g_2(z) = \frac{z^2}{(z-i)^2}$,
\[
g_2'(z) = \frac{(z-i)^2(2z)-2(z-i)z^2}{(z-i)^4},\quad g_2'(-i) = \frac{(-2i)^2(2i)-2(-2i)(-2i)^2}{(-2i)^4} = \frac{i}{4}.
\]
Hence
\[
\Res (f;-i) = \frac{g_2'(i)}{1!} = \frac{i/4}{1} = \frac{i}{4}.
\]
\part Pole of order $2$ at $z=i/4$.  Write $f(z) = \dfrac{\Log(z)}{16 (z-\frac{i}{4})^2}$, then
\[
\Res(f;i/4) = \frac{1}{16} \cdot \frac{1}{(i/4)} = - \frac{i}{4}.
\]
\part Use the $g/h$ rule with $g(z)=1$ and $h(z)=\exp(z)-1$.  Then $h(z)=0$ whenever $\exp(z)=1$, i.e. at the points $z_k = 2\pi i k$ where $k \in \mathbb{Z}$, so $f$ has isolated singularities at these points.  Since $h'(z) = \exp(z),\ h'(z_k)=1$ for all $k$, and so each $z_k$ is a pole of order $1$ for the function $f$.  By the $g/h$ rule,
\[
\Res (f;z_k) = \frac{g(z_k)}{h'(z_k)} = 1.
\]
\end{parts}


\end{answer}
\question Evaluate
\[
\int_{\mathcal{C}} \frac{1}{z(z-1)(z+2)}\ dz,
\]
where $\mathcal{C}$ is the anticlockwise circle with centre $0$ and radius $3/2$.
\begin{answer}
The function $f$ defined by
\[
f(z) = \frac{1}{z(z-1)(z+2)}
\]
has isolated singularities at $0,1$ and $2$, and the first two of these are enclosed by $\mathcal{C}$.  The residues are
\begin{align*}
\Res (f;0) &= \frac{1}{(0-1)(0+2)}  = - \frac{1}{2} \\
 \Res (f;1) & = \frac{1}{1(1+2)}  = \frac{1}{3}.
\end{align*}
Hence by the Residue Theorem
\[
\int_{\mathcal{C}} \frac{1}{z(z-1)(z+2)}\ dz = 2\pi i \left( - \frac{1}{2} + \frac{1}{3} \right) = - \frac{i\pi}{3}.
\]
\end{answer}
\question Evaluate
\[
\int_{\mathcal{C}} \frac{1}{(z^2+1)^3}\ dz
\]
where $\mathcal{C}$ is the anticlockwise square with vertices $1,\ 1+2i, -1+2i$ and $-1$.
\begin{answer}
Using the factorisation $(z^2+1)^3=(z+i)^3(z-i)^3$, we see that this function, call it $f$, has poles of order $2$ at $z=\pm i$.  Of these, only the pole at $z=i$ is enclosed by $\mathcal{C}$, so that 
\[
\int_{\mathcal{C}} \frac{1}{(z^2+1)^3}\ dz = 2\pi i \Res (f;i).
\]
If we let $g(z) = \dfrac{1}{(z+i)^3}$ then
\[
f(z) = \frac{g(z)}{(z-i)^3},
\]
$g'(z) = -3 (z+i)^{-4}$ and $g''(z) = 12(z+i)^{-5}$.  Hence
\[
\Res (f;i) = \frac{g''(i)}{2!} = \frac{12}{2(2i)^5} = \frac{3}{16i}
\]
which gives
\[
\int_{\mathcal{C}} f = 2\pi i \left( \frac{3}{16i} \right) = \frac{3\pi}{8}.
\]
\end{answer}
\question Use contour integration to evaluate each of the following real integrals:
\begin{parts}
\part \hfil$\displaystyle
\int_0^{2\pi} \frac{1}{5+4 \sin \theta}\ d\theta 
$\hfil
\part \hfil$\displaystyle \int_0^{\infty} \frac{1}{x^4+1}\ dx$\hfil
\end{parts}
\begin{answer}
\begin{parts}
\part We use the fact that for a rational function $R$ of two (real) variables,
\[
\int_0^{2\pi} R( \cos(t), \sin (t) ) dt = \int_{\mathcal{C}} f
\]
where $\mathcal{C}$ is the anticlockwise unit circle and $f$ is the complex function defined by
\[
f(z) = R \left( [z+z^{-1}]/2 , [z-z^{-1}]/(2i) \right) \cdot \frac{1}{iz}.
\]
In this case, we have
\begin{align*}
f(z) & = \frac{1}{5+4[z-z^{-1}]/2i} \cdot \frac{1}{iz} \\
& = \frac{1}{2z^2+5iz-2} \\
& = \frac{1}{(2z+i)(z+2i)}.
\end{align*}
Then with $\mathcal{C}$ the anticlockwise unit circle, the only singularity of $f$ enclosed by $\mathcal{C}$ is at $z=-i/2$, and so
\begin{align*}
\int_0^{2\pi} \frac{1}{5+4 \sin \theta}\ d\theta &= \int_{\mathcal{C}} \frac{1}{(2z+i)(z+2i)}\ dz \\
& = 2\pi i \Res (f;-i/2) \\
& = 2\pi i \cdot \frac{1}{2(-i/2+2i)} = \frac{2\pi}{3}.
\end{align*}
\part 
We shall consider the complex function
\[
f(z) = \frac{1}{z^4+1}
\]
which is holomorphic everywhere in $\C$ except where $z^4=-1$; that is to say, $f$ is holomorphic on $\C \backslash \set{ e^{i\pi/4},\ e^{i3\pi/4},\ e^{i 5\pi/4},\ e^{i 7\pi/4} }$.  Let $\mathcal{C}_R = L_R+S_R$ where $R>1$, $L_R$ is the straight line path $[-R,R]$ and $S_R$ is the upper-semicircle with centre $0$ and radius $R$ from $R$ to $-R$ via $iR$; then the poles of $f$ enclosed by $\mathcal{C}_R$ are at $e^{i\pi/4}$ and $e^{i3\pi/4}$.

With $g(z)=1$ and $h(z)=z^4+1$, we have $h'(z)=4z^3$ so that by the $g/h$ rule,
\begin{align*}
\Res(f;e^{i\pi/4}) &= \frac{1}{4(e^{i\pi/4})^3} = \frac{1}{4e^{3\pi/4}} = \frac{1}{4(-a+ia)} \\
\Res(f;e^{i3\pi/4}) &= \frac{1}{4(e^{i3\pi/4})^3} = \frac{1}{4e^{9\pi/4}} = \frac{1}{4(a+ia)} \\
\end{align*}
where $a=1/\sqrt{2}$. So (for $R>1$) we have
\begin{align*}
\int_{\mathcal{C}_R} f &= 2\pi i \left( \text{ sum of residues at poles of $f$ inside $\mathcal{C}_R$} \right) \\
& = 2\pi i \left( \frac{1}{4(-a+ia)}+\frac{1}{4(a+ia)}) \right)
& = \frac{\pi}{\sqrt{2}}
\end{align*}
by the Residue Theorem.

We now show that 
\[
\lim_{R \to \infty} \int_{S_R} f =0
\]
using the Estimation Lemma.  Indeed, for $z \in S_R$, $\abs{z}=R$ and so by the reverse triangle inequality
\[
\abs{z^4+1} \geq \abs{ \abs{z^4} - \abs{1} } = \abs{ \abs{z}^4-1 } = R^4-1.
\]
Hence for all such $z$,
\[
\abs{ f(z) } = \abs{\frac{1}{z^4+1}} = \frac{1}{\abs{z^4+1}} \leq \frac{1}{R^4-1}.
\]
The Estimation Lemma then gives
\[
\abs{ \int_{S_R} f } \leq \frac{1}{R^4-1} \cdot \pi R = \frac{\pi R}{R^4-1}
\]
so that
\[
\int_{S_R} f \to 0 \quad\text{ as }\quad R \to \infty.
\]

Using the parametrisation $\gamma_L:[-R,R] \to \C$, $\gamma(t)=t$ of $L_R$, we have
\[
\int_{L_R} f = \int_{-R}^R \frac{1}{(t^4+1}\ dt.
\]
Hence
\begin{align*}
\frac{\pi}{\sqrt{2}} & = \int_{\mathcal{C}_R} f \\
& = \lim_{R \to \infty} \int_{\mathcal{C}_R} f \\
& = \lim_{R \to \infty} \int_{-R}^R \frac{1}{t^4+1}\ dt + \lim_{R \to \infty} \int_{S_R} f \\
& = \int_{-\infty}^{\infty} \frac{1}{t^4+1}\ dt.
\end{align*}
Since the integrand is even, it follows that
\[
\int_{0}^{\infty} \frac{1}{t^4+1}\ dt = \frac{1}{2} \int_{-\infty}^{\infty} \frac{1}{t^4+1}\ dt = \frac{\pi}{2\sqrt{2}}.
\]

\end{parts}
\end{answer}
\question Use contour integration to evaluate the following real integrals:
\begin{parts}
\part \[
\int_0^{2\pi} \frac{1}{16 \cos^2 (t)+25 \sin^2 (t)}\ dt.
\]
\begin{answer}
As before, we use the fact that for a rational function $R$ of two (real) variables,
\[
\int_0^{2\pi} R( \cos(t), \sin (t) ) dt = \int_{\mathcal{C}} f
\]
where $\mathcal{C}$ is the anticlockwise unit circle and $f$ is the complex function defined by
\[
f(z) = R \left( [z+z^{-1}]/2 , [z-z^{-1}]/(2i) \right) \cdot \frac{1}{iz}.
\]
Here $f$ is given by
\begin{align*}
f(z) = \frac{4iz}{9z^4-82z^2+9},
\end{align*}
and so
\[
\int_{\mathcal{C}} \frac{4iz}{9z^4-82z^2+9}\ dz
\]
where $\mathcal{C}$ is the anticlockwise unit circle.  Factorising the denominator gives
\[
9z^4-82z^2+9 = (9z^2-1)(z^2-9) = (3z-i)(3z+i)(z-3i)(z+3i),
\]
and so the poles of $f$ enclosed by $\mathcal{C}$ are at $z= \pm i/3$.  Hence by the Residue Theorem
\[
\int_{\mathcal{C}} f = 2\pi i \left[ \Res (f;i/3)+ \Res (f;-i/3) \right] = \frac{\pi}{10}
\]
and so
\[
\int_0^{2\pi} \frac{1}{16 \cos^2 (t)+25 \sin^2 (t)}\ dt = \frac{\pi}{10}.
\]


\end{answer}
\part
\[
\int_{-\infty}^{\infty} \frac{1}{(x^2+1)(x^2+9)}\ dx.
\]
\begin{answer}
With $\displaystyle f(z)=\frac{1}{(z^2+1)(z^2+9)} = \frac{1}{z^4+10z+9}$, $R>3$ and $\mathcal{C}_R=L_R+S_R$ as before, the Residue Theorem gives
\[
\int_{\mathcal{C}_R} f = 2\pi i \left( \Res(f;i)+\Res(f;3i) \right) = \frac{\pi}{12}.
\]
The reverse triangle inequality gives
\[
\abs{\frac{1}{(z^2+1)(z^2+9)}} \leq \frac{1}{(R^2-1)(R^2-9)}
\]
for all $z \in S_R$ so that by the Estimation Lemma
\[
\abs{ \int_{S_R} f } \leq \frac{\pi R}{(R^2-1)(R^2-9)} \to 0 \quad\text{as}\quad R \to \infty.
\]
A similar argument to the one used before yields
\[
\int_{-\infty}^{+\infty} \frac{1}{(x^2+1)(x^2+9)}\ dx = \frac{\pi}{12}.
\]
\end{answer}
\part \[
\int_0^{\infty} \frac{\cos(5x)}{x^2+4}
\]
(Hint: first use the usual method to evaluate
$
\displaystyle \int_{-\infty}^{+\infty} \frac{e^{i5x}}{x^2+4}\ dx.
$)
\begin{answer}
Following the hint, we let $f(z) = \dfrac{\exp(5iz)}{z^2+4}$, and consider the contour $\mathcal{C}_R=L_R+S_R$ as before, where $R>2$.  Using the Residue Theorem we get
\[
\int_{\mathcal{C}_R} f = 2 \pi i  \Res (f;2i) = \frac{\pi}{2}e^{-10}.
\]
For $z=x+iy \in S_R$ we have
\begin{align*}
\abs{\exp(5iz)} &= \abs{\exp (5i(x+iy))} \\
& = \abs{\exp(-5y+5ix)} \\
& = \abs{\polar{e^{-5y}}{5x}} \\
& = \abs{e^{-5y}} \abs{\cos(5x)+i\sin(5x)} = e^{-5y}.
\end{align*}
Since $S_R$ lies above the real axis, $x+iy \in S_R$ implies $y\geq 0$, so that $-5y \leq 0$ and so $e^{-5y} \leq e^0=1$.  Thus $\abs{\exp(5iz)} \leq 1$ for all $z \in S_R$.

By the reverse triangle inequality $\abs{z^2+4} \geq R^2-4$ for all $z \in S_R$, thus for all such $z$,
\[
\abs{f(z)} \leq \frac{1}{R^2-4}.
\]
Together with the Estimation Lemma we see that
\[
\abs{\int_{S_R} f } \leq \pi R \cdot \frac{1}{R^2-4} \to 0 \quad\text{as}\quad R \to \infty.
\]
Arguing as before, we get
\[
\int_{-\infty}^{+\infty} \frac{e^{5ix}}{x^2+4}\ dx = \frac{\pi}{2}e^{-10}.
\]
Now, for all $x \in \R$, $e^{5ix} = \cos(5x)+i\sin(5x)$, so that
\[
\int_{-\infty}^{+\infty}  \frac{e^{5ix}}{x^2+4}\ dx = \int_{-\infty}^{+\infty} \frac{\cos(5x)}{x^2+4}\ dx \ + i \int_{-\infty}^{+\infty} \frac{\sin(5x)}{x^2+4}\ dx
\]
or in other words
\[
\int_{-\infty}^{+\infty} \frac{\cos(5x)}{x^2+4}\ dx = \Re \left( \int_{-\infty}^{+\infty}  \frac{e^{5ix}}{x^2+4}\ dx \right) = \frac{\pi}{2} e^{-10}.
\]
Finally, since the integrand is even,
\[
\int_0^{\infty} \frac{\cos(5x)}{x^2+4}\ dx = \frac{1}{2} \left( \int_{-\infty}^{+\infty} \frac{\cos(5x)}{x^2+4}\ dx \right) = \frac{\pi}{4} e^{-10}.
\]
\end{answer}
\end{parts}
\question
\begin{parts}
\part Let $N$ be a natural number and let $\alpha_j$ be constants for $-N \leq j \leq N$.  If $f(z) = \displaystyle \sum_{j=-N}^N \alpha_j z^j$, write down the value of $\Res(f;0)$.
\part Write $\displaystyle \int_0^{2\pi} \left[ \cos(t) \right]^{8}\ dt$ as a contour integral, and use part (a) to evaluate it.
\end{parts}
\begin{answer}
Let $\mathcal{C}$ be the anticlockwise circle with centre $0$ and radius $1$, so that
\[
\Res (f;0) = \frac{1}{2\pi i} \int_{\mathcal{C}} f
\]
We know that for $n \neq -1$, the function $z \mapsto z^n$ has an antiderivative on $\C \backslash \set{0}$, hence  $\displaystyle \int_{\mathcal{C}} z^n\ dz =0$ for $n \neq -1$.  It follows that
\begin{align*}
\Res (f;0)  = \frac{1}{2\pi i}\int_{\mathcal{C}} f & = \frac{1}{2\pi i} \int_{\mathcal{C}} \sum_{j=-N}^N \alpha_j z^j\ dz \\
& = \frac{1}{2\pi i} \sum_{j=-N}^N \alpha_j \int_{\mathcal{C}} z^j\ dz \\
& = \frac{1}{2\pi i} \alpha_{-1} \int_{\mathcal{C}}  z^{-1}\ dz \\
& =  \frac{1}{2\pi i} \cdot \alpha_{-1} \cdot 2\pi i = \alpha_{-1}.
\end{align*}
For (b), following previous examples we know that
\[
\int_0^{2\pi} \left( \cos(t) \right)^{8}\ dt = \int_{\mathcal{C}} \left( \frac{z+z^{-1}}{2} \right)^8 \cdot \frac{1}{iz}\ dz
\]
The binomial formula allows us to expand
\[
(z+z^{-1})^8 = \sum_{k=0}^8\binom{8}{k} z^k(z^{-1})^{8-k} = \sum_{k=0}^8\binom{8}{k} z^{2k-8},
\]
hence
\[
\frac{1}{iz} \left( \frac{z+z^{-1}}{2} \right)^8 = \frac{1}{i2^8} \sum_{k=0}^8\binom{8}{k} z^{2k-9}.
\]
The only isolated singularity of this function is at $z=0$, and by part (a), $\Res(f;0)$ is simply the coefficient of $z^{-1}$, i.e. $\frac{1}{i2^8}\binom{8}{4} =\frac{35}{128i}$.  By the Residue Theorem,
\[
\int_{\mathcal{C}} f = 2\pi i \frac{35}{128i} = \frac{35\pi}{64}.
\] 
\end{answer}

\end{questions}