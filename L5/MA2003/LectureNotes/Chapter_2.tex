% !TEX root = main.tex

\chapter[Holomorphic Functions]{Differentiability and Holomorphic Functions}
\section{The Derivative of a Complex Function}
Throughout this section, let $U$ be an open subset of $\C$.  The definition of differentiability for a function $f: U \to \C$ looks almost identical to the corresponding definition for real functions.

\begin{definition}
Let $f: U \to \C$ be a function, and let $z \in U$.  We say that $f$ is \emph{differentiable at $z$} if the limit
\begin{equation}
\label{e:diff}
\lim_{\substack{h \to 0\\ h \in \C \backslash \set{0}}} \frac{f(z+h)-f(z)}{h}
\end{equation}
exists.  When it does, we denote its value by $f'(z)$.

%If $f$ is differentiable at $w$ for all $w \in U$, we say that $f$ is \emph{holomorphic on $U$}.
\end{definition}

Since $U$ is open, for a fixed $z \in U$, the quantity (the \emph{difference quotient})
\[
\frac{f(z+h)-f(z)}{h}
\]
is defined whenever $h$ is `sufficiently small' (to ensure that $z+h \in U$ and so $f(z+h)$ is defined) but not zero (to avoid division by zero). In other words, $0$ is a limit point of the domain of the difference quotient (regarded as a function of $h$).

 For this reason, we can again omit the second subscript and write~\eqref{e:diff} as
\[
\lim_{h \to 0} \frac{f(z+h)-f(z)}{h}
\]
without ambiguity.




\begin{example}
\label{e:diff1}
Let us investigate whether or not the function
\[
f:\C \to \C, \quad f(z)=-i\conj{z}
\]
is differentiable at any points $z \in \C$.
\end{example}

%1 page handwritten.
\begin{solution}
Fix $z \in \C$. To evaluate the limit
\[
\lim_{h \to 0} \frac{f(z+h)-f(z)}{h},
\]
we first try to simplify the expression
\[
\frac{f(z+h)-f(z)}{h}.
\]
For any $z \in \C \backslash \set{0}$, 
\[
\frac{f(z+h)-f(z)}{h} = \frac{-i\ \conj{(z+h)}-(-i)\conj{z}}{h} = -i \frac{\conj{h}}{h}.
\]
%\vspace*{8cm}
But then the limit
\[
\lim_{h \to 0} \frac{f(z+h)-f(z)}{h} = \lim_{h \to 0} -i \frac{\conj{h}}{h},
\]
is the same as the limit that we considered in Example~\ref{e:rlim}, except scaled by a factor of $-i$.  Hence this limit does not exist for the same reason; that is
\[
\rlim{h \to 0}{h \in \R \backslash \set{0}} -i \frac{\conj{h}}{h} = -i, \text{ while } \rlim{h \to 0}{h \in i\R \backslash \set{0}} -i \frac{\conj{h}}{h} = i.
\]
So $f$ is not differentiable at any point $z \in \C$.
\end{solution}



\begin{example}
\label{e:diff2}
The function $\mathbf{g}: \R^2 \to \R^2$ is defined by
\[
\mathbf{g}(x,y)=(1,-x^2-y^2)\quad (x,y) \in \R^2.
\]
Find the corresponding complex function $g:\C \to \C$ and investigate where it is differentiable.
\end{example}

\begin{solution}
The corresponding function $g: \C \to \C$ is
\[
g(x+iy) = 1+i(-x^2-y^2),
\]
or equivalently,
\[
g(z) = 1-i z \conj{z}.
\]
Again, we fix $z \in \C$ and look at the difference quotient
\[
\frac{g(z+h)-g(z)}{h} = -i(z \frac{\conj{h}}{h}+\conj{z}+\conj{h})
\]
for $h \in \C \backslash \set{0}$.\\

As with the previous example, we have $\frac{\conj{h}}{h}$ appearing in the difference quotient, so it looks like our limit will not exist.  Again, we shall check by evaluating the restricted limits along the real and imaginary axes.

\begin{align*}
\rlim{h \to 0}{h \in \R \backslash \set{0}} -i \left( z \frac{\conj{h}}{h} + \conj{z} + \conj{h} \right) & = -i(z + \conj{z} ) \\
\rlim{h \to 0}{h \in i\R \backslash \set{0}} -i \left( z \frac{\conj{h}}{h} + \conj{z} + \conj{h} \right) & = -i (-z+\conj{z} ).
\end{align*}
These restricted limits are equal if and only if $z+\conj{z} = -z + \conj{z}$, which occurs if and only if $z=0$.  It follows that the unrestricted limit
\[
\lim_{h \to 0} \frac{g(z+h)-g(z)}{h}
\]
does not exist at any $z \in \C \backslash \set{0}$, hence $g$ is not differentiable at any of these points.

What about when $z=0$?  While the restricted limits along the real and imaginary axes are equal, this is \emph{not} enough to conclude that the unrestricted limit exists.  Indeed, we must verify this directly:
\[
\lim_{h \to 0} \frac{g(0+h)-g(0)}{h} = \lim_{h \to 0} -i \conj{h} = 0.
\]
(Here we have used the fact that $z \mapsto \conj{z}$ is continuous (Example~\ref{e:cts}), and so limit can be found by substitution of $z=0$.)  In other words, $g$ is differentiable at $0$ (and nowhere else), with $g'(0)=0$.
%\vspace*{15cm}
\end{solution}
\begin{note}
In Example~\ref{e:diff2}, both the real and the imaginary parts of $g$ are differentiable with respect to both $x$ and $y$ everywhere in $\R^2$.
\end{note}


\begin{example}
\label{e:diff3}
Consider the function $f$ defined by $f(z)=\dfrac{1}{z}$.  At which points $z \in \C$ is $f$ differentiable?
\end{example}

\begin{solution}
Since the domain of $f$ is the (open) set $\C \backslash \set{0}$, we must necessarily restrict our attention to points $z$ in this set.  Then with $h \neq 0$, we have
\[
\frac{f(z+h)-f(z)}{h} = \frac{\left( \frac{1}{z+h} \right) - \frac{1}{z}}{h} = \frac{-1}{z(z+h)}.
\]
Since by assumption $z \neq 0$, the algebra of limits (Proposition~\ref{p:alglimits}) tells us that $-\dfrac{1}{z(z+h)} \to - \dfrac{1}{z^2}$ as $h \to 0$.  Thus $f$ is differentiable at every $z \in \C \backslash \set{0}$, with $f'(z) = -\dfrac{1}{z^2}$.
%\vspace*{6cm}
\end{solution}

\begin{example}
\label{e:diff4}
Fix $\alpha \in \C$ and let $f$ be the constant function $f(z) = \alpha$ for all $z \in \C$.  Then $f$ is differentiable at every point $z \in \C$, and satisfies $f'(z)=0$ for all $z \in \C$.
\end{example}
\begin{solution}
For any $z\in \C$, and $h \in \C \backslash \set{0}$,
\[
\frac{f(z+h)-f(z)}{h} = \frac{\alpha-\alpha}{h}=0 \to 0 \text{ as } h \to 0.
\]
\end{solution}

\begin{question}
If $f:U \to \C$ is differentiable at every point $z$ in $U$, and has $f'(z)=0$ for all $z$, is $f$ necessarily constant?
\end{question}

\begin{answer}
Not necessarily.  If $U=U_1 \cup U_2$ where $U_1$ and $U_2$ are disjoint open subsets of $\C$, and $f:U \to \C$ is defined by
\[
f(z) = \begin{cases}
0 & \text{ if } z \in U_1 \\
1& \text{ if } z \in U_2,
\end{cases}
\]
then $f'(z)$ exists and is equal to $0$ at every $z \in U$, but $f$ is non-constant.
\end{answer}
%\vspace*{5cm}

\begin{example}
\label{e:diff5}
Fix $\beta \in \C$ and let $f(z)=\beta z$.  Then $f'(z_0) = \beta$ for all $z_0 \in \C$.
\end{example}

\begin{comment}
\begin{solution}
\[
\frac{f(z+h)-f(z)}{h} = \frac{\beta(z+h)-\beta z}{h}=\beta  \to 0 \text{ as } h \to 0.
\]
\end{solution}
\end{comment}

\section{Holomorphic Functions}

\begin{definition}
Let $U \subseteq \C$ be an open set and $f:U \to \C$ a complex function.  If $f$ is differentiable at $z$ for all $z \in U$, we say that $f$ is \emph{holomorphic on $U$}.
\end{definition}
Why \emph{holomorphic} and not simply \emph{differentiable} on $U$?  One reason for this is that there are many functions $f$ that are differentiable on the real line, but fail to be differentiable throughout any open subset $U \subseteq \C$.


\begin{definition}
Let $f:U \to \C$ be a complex function and $z \in U$.  If there is some $r>0$ such that $f$ is differentiable at every point in the open disc $D(z,r)$, then $f$ is said to be \emph{holomorphic at $z$}.
\end{definition}

If a function $f$ is holomorphic at $z$, then it is differentiable at $z$, but the converse is not necessarily true.


\begin{example}
Let us examine whether or not the functions from Examples~\ref{e:diff1},~\ref{e:diff2} and~\ref{e:diff3} are holomorphic on any subsets of $\C$.
\begin{blankbox}
\begin{itemize}
\item $f(z)=-i\conj{z}$ is not differentiable at any $z \in \C$ and therefore not holomorphic on any $U \subseteq \C$.
%\vspace*{2cm}
\item $f(z) = 1-i z \conj{z}$ is differentiable at $0$ and nowhere else.  Since the one-point set $\set{0}$ is not an open set, there is no open subset $U \subseteq \C$ with $f$ holomorphic on $U$.
%\vspace*{3cm}
\item $f(z) = \dfrac{1}{z}$ is differentiable at every point of the (open) set $\C \backslash \set{0}$ and is therefore holomorphic on this set.
%\vspace*{3cm}
\end{itemize}
\end{blankbox}
\end{example}




If $f:U \to \C$ is holomorphic on $U$, then we get a new function $f':U \to \C, \ z \mapsto f'(z)$, called the \emph{derivative of $f$}.  In other words, for $z \in U$, $f'(z)$ is defined as the limit
\[
f'(z) = \lim_{\substack{h \to 0 \\ h \in \C \backslash \set{0}}} \frac{f(z+h)-f(z)}{h}.
\]



As for differentiable functions in $\R$, we have sum, product, chain and quotient rules for holomorphic functions defined on open subsets of $\C$.

\begin{theorem}[Rules of Differentiation; proof non-examinable]
\label{t:diffrules}
Let $U$ and $V$ be open subsets of $\C$ and let $f:U \to \C$ and $g:V \to \C$ be holomorphic on $U$ and $V$ respectively.  Then
\begin{enumerate}
\item (Sum rule) $f+g$ is homomorphic on $U \cap V$ and $(f+g)'(z)=f'(z)+g'(z)$
\item (Scalar Multiples) For any $\alpha \in \C$, $(\alpha f )$ is holomorphic on $U$ and $(\alpha f)' (z) = \alpha f'(z)$
\item (Product Rule) $fg$ is holomorphic on $U \cap V$ and $(fg)'(z)=f'(z)g(z)+g'(z)f(z)$
\item (Quotient Rule) The quotient $f/g$ is holomorphic on $U \cap \set{ z \in V: g(z) \neq 0}$ and 
\[
\left( \frac{f}{g} \right) '(z) = \frac{f'(z)g(z)-f(z)g'(z)}{g(z)^2}
\]
\item (Chain Rule) $f \circ g$ is holomophic on $V \cap g^{-1}(U)$ and $(f\circ g)'(z) = f'(g(z))g'(z).$
\end{enumerate}
\end{theorem}

%\vspace*{7cm}
\begin{comment}
Since we only ever speak of a function being holomorphic on an \emph{open} subset $U \subseteq \C$, the statement of Theorem~\ref{t:diffrules} implicitly relies on the assumption that $U \cap V$, $U \cap \set{ z \in V: g(z) \neq 0 }$ and $V \cap g^{-1} (U)$ are all open whenever $U$ and $V$ are open and $f$ and $g$ are holomorphic.  These are all relatively easy to prove (the second and third rely on Proposition~\ref{p:diffimpliescontinuous}).

The proof of Theorem~\ref{t:diffrules} is very similar to that of the corresponding result for functions of a real variable, and is thus omitted.
\end{comment}
\begin{example}
 Let $p:\C \to \C$ be a complex polynomial
\[
p(z) = \alpha_0 + \alpha_1z + \ldots + \alpha_n z^n.
\]
Then $p$ is holomorphic on $\C$.
\end{example}

\begin{solution}
We saw in Examples~\ref{e:diff4} and~\ref{e:diff5} that the functions $f(z) = \beta z$ and $g(z)= \alpha$ (where $\alpha, \beta \in \C$ are fixed), are holomorphic on $\C$ with derivatives $f'(z) = \beta$ and $g'(z)=0$ for all $z \in \C$.  Together with Theorem~\ref{t:diffrules}, this shows that $p(z)$ is holomorphic on $\C$ with derivative
\[
p'(z) = \alpha_1 + 2\alpha_2 z + \ldots + n \alpha_n z^{n-1}.
\]
Thus complex polynomials can be differentiated using exactly the same rules as for real polynomials
\end{solution}
\begin{note}
A similar argument shows that if $g(z) = \dfrac{1}{z^n}$ where $n >0$, then $g'(z) = -\dfrac{n}{z^{n+1}}$.
\end{note}
\begin{example}
The complex functions $\exp,\ \sin$ and $\cos$ defined in Chapter 1 are also holomorphic on $\C$, with derivatives
\[
\left( \exp (z) \right)'=\exp(z) \quad \left( \sin (z) \right)' = \cos (z) \quad\text{and}\quad \left( \cos(z) \right)' = - \sin(z).
\]
\end{example}

%\vspace*{6cm}


\begin{proposition}[Proof non-examinable]
\label{p:diffimpliescontinuous}
Let $f: U \to \C$ be differentiable at a point $z \in U$.  Then $f$ is continuous at $z$.
\end{proposition}


\begin{question} If $f$ is holomophic on $U$, is $f'$ holomorphic on $U$?
\end{question}
\begin{example}
In real analysis, consider $g:\R \to \R$ defined by
\[
g(x) = \begin{cases}
-x^2 & x<0 \\
x^2 & x\geq 0.
\end{cases}
\]
\begin{blankbox}
It is easily shown that $g$ is differentiable on $\R$ with $g'(x) = 2 \abs{x}$ for all $x \in \R$.  However, we know that $x \mapsto 2 \abs{x}$ fails to be differentiable at $0$.

In fact for holomorphic functions, the answer is yes.  This gives us an even stronger result: if $f:U \to \C$ is holomorphic on $U$, then $f$ is infinitely differentiable on $U$.  We shall return to this later on in the module.
\end{blankbox}
\end{example}

\section{The Cauchy Riemann Equations}
Again, let $U \subseteq \C$ be open.  Suppose we are given $f:U \to \C$, then we have seen how to write $f$ as a sum of its real and imaginary parts:


\[ f(x+iy)=u(x,y)+iv(x,y) \]
for all $z=x+iy \in U$.  \

\begin{comment}
Written in terms of $z$, we have
\[
f(z) = \underbrace{\frac{f(z)+\conj{f(z)}}{2}}_{\text{Real}} + i \underbrace{\left(\frac{f(z)-\conj{f(z)}}{2i} \right)}_{\text{Real}}
\]
\end{comment}


\begin{question}
If $u$ and $v$ are differentiable with respect to both $x$ and $y$, does it follow that $f$ is holomorphic?  
\end{question}

%\vspace*{2cm}
\begin{answer}
No.  We saw in Example~\ref{e:diff1} that
\[
f(x+iy) = 1-i (x^2+y^2),
\]
whose real and imaginary parts are differentiable everywhere in $\R^2$ with respect to both $x$ and $y$, is not differentiable at any point $z \in \C \backslash \set{0}$.
\end{answer}

For clarity, let us recall the definition of partial derivatives.



\begin{definition}
Suppose that $u: \R^2 \to \R$ is a function.  Then the \emph{partial derivatives} of $u$ with respect to $x$ and $y$ are the functions $\frac{\partial u}{\partial x}$ and $\frac{\partial u}{\partial y}$ respectively, defined via
\begin{align*}
\frac{\partial u}{\partial x} (x_0,y_0):&= \lim_{\substack{h\to 0\\h \in \R \backslash \set{0}}} \frac{u(x_0+h,y_0)-u(x_0,y_0)}{h}\\
\frac{\partial u}{\partial y} (x_0,y_0):&= \lim_{\substack{h\to 0\\h \in \R \backslash \set{0}}} \frac{u(x_0,y_0+h)-u(x_0,y_0)}{h}
\end{align*}
at the points $(x_0,y_0)$ where the limits exist.
\end{definition}



\begin{theorem}[Differentiability Implies the Cauchy-Riemann Equations]
\label{t:cr1}

Let $f$ be a complex-valued function defined on some open set $U$ and let $z_0=x_0+iy_0 \in U$.  Write
\[
f(x+iy) = u(x,y)+iv(x,y).
\]
Then if $f$ is differentiable at the point $z_0$ we have the following:

\begin{enumerate}
\item[(i)] The partial derivatives $\pd{u}{x}, \pd{u}{y}, \pd{v}{x}, \pd{v}{y}$ all exist at the point $(x_0,y_0) \in \R^2$ corresponding to $z_0$.
\item[(ii)] The partial derivatives satisfy the Cauchy-Riemann Equations at $(x_0,y_0)$:
\begin{equation}
\label{eq:cr}
\pd{u}{x} (x_0,y_0) = \pd{v}{y} (x_0,y_0),\quad \pd{u}{y} (x_0,y_0) = - \pd{v}{x} (x_0,y_0)
\end{equation}
\item[(iii)] At this point  the derivative of $f$ satisfies
\[
f'(z_0)= \pd{u}{x} (x_0,y_0) + i \pd{v}{x} (x_0,y_0) = \pd{v}{y} (x_0,y_0) - i \pd{u}{y} (x_0,y_0).
\]
\end{enumerate}
\end{theorem}

Before proving Theorem~\ref{t:cr1}, let us look at some examples that demonstrate how it is a powerful result.



\begin{example}
 Verify that the Cauchy-Riemann equations are satisfied by the function  $f(z)=z^2$.\\
\end{example}
\begin{solution}
Here we have
\[
f(x+iy) = \underbrace{x^2-y^2}_{u(x,y)} + i \underbrace{2xy}_{v(x,y)},
\]
and hence
\[
\pd{u}{x} = 2x = \pd{v}{y} \text{ and } \pd{u}{y} = -2y = - \pd{v}{x}.
\]
Thus~\eqref{eq:cr} holds at every $(x,y) \in \R^2$.  Moreover, we have
\[
\pd{u}{x} + i \pd{v}{x} = 2x+i2y = 2(x+iy) = 2z = f'(z),
\]
as expected.
\end{solution}
\begin{example} Use the Cauchy-Riemann equations to investigate the differentiability of $f(z)=\conj{z}$.
\end{example}

\begin{solution}
This time
\[
f(x+iy) = \underbrace{x}_{u(x,y)} + i \underbrace{-y}_{v(x,y)},
\]
so that $\pd{u}{x} = 1$ while $\pd{v}{y}=-1$ and~\eqref{eq:cr} does not hold at any point $(x,y) \in \R^2$ (note that we need both of the equations in~\eqref{eq:cr} to hold).
 Thus we conclude that $f(z) = \conj{z}$ is not differentiable at any point $ z \in \C$.
 \end{solution}

\begin{exercise}
Before we cover it in lectures, try to prove Theorem~\ref{t:cr1} yourself as follows:
\begin{enumerate}
\item[(i)] Assuming that $f$ is differentiable at $z_0$, what can be said about the restricted limits
\[
\rlim{h \to 0}{h \in \R \backslash \set{0}} \frac{f(z_0+h)-f(z_0)}{h} \quad\text{and}\quad \rlim{h \to 0}{h \in \C \backslash \set{0}} \frac{f(z_0+h)-f(z_0)}{h} \quad ?
\]
\item[(ii)] Write the restricted limits from (i) in terms of (the partial derivatives of) $u$ and $v$.  The result should follow easily.
\end{enumerate}
\end{exercise}
\begin{proof}[Proof of Theorem~\ref{t:cr1}]
Since $f$ is differentiable at $z_0$ we know that
\[
\rlim{h\to 0}{h \in \C \backslash \set{0}} \frac{f(z_0+h)-f(z_0)}{h}
\]
exists and is equal to $f'(z_0)$.  In particular, all of corresponding restricted limits exist, and are also equal to $f'(z_0)$.  As before, we shall examine the restricted limits along the real and imaginary axes.

If we restrict $h$ to the nonzero real axis, then
\[
z_0+h = (x_0+h)+iy_0,
\]
which corresponds to the point $(x_0+h,y_0) \in \R^2$.  Thus the restricted limit satisfies:

%\vspace*{10cm}
\begin{align*}
f'(z_0) & = \rlim{h \to 0}{h \in \R \backslash \set{0}} \frac{f(z_0+h)-f(z_0)}{h} \\
& = \rlim{h \to 0}{h \in \R \backslash \set{0}} \frac{\left[ u(x_0+h,y_0)+iv(x_0+h,y_0) \right] - \left[ u(x_0,y_0)+iv(x_0,y_0) \right]}{h} \\
& \vspace*{2cm} \\
& = \left( \rlim{h \to 0}{h \in \R \backslash \set{0}} \frac{u(x_0+h,y_0)-u(x_0,y_0)}{h}  \right) +i \left(  \rlim{h \to 0}{h \in \R \backslash \set{0}} \frac{v(x_0+h,y_0)-v(x_0,y_0)}{h} \right)  \\
& = \pd{u}{x} (x_0,y_0)+i \pd{v}{x} (x_0,y_0).
\end{align*}
Hence both $\pd{u}{x}$ and $\pd{v}{x}$ exist at the point $(x_0,y_0)$, and the derivative of $f$ at $z_0$ satisfies
\begin{equation}
\label{eq:cr1}
\tag{$\dagger$}
f'(z_0) = f'(x_0+iy_0) = \pd{u}{x} (x_0,y_0)+i \pd{v}{x} (x_0,y_0).
\end{equation}
%\vspace*{7cm}

We shall now examine the restricted limit along the nonzero imaginary axis. This limit must also satisfy
\[
f'(z_0) = \rlim{h\to 0}{h \in i \R \backslash \set{0}} \frac{f(z_0+h)-f(z_0)}{h}.
\]
%\vspace*{3cm}
If $h \in i \R \backslash \set{0}$, then $h=0+ik$ for some $k \in \R \backslash \set{0}$, and so
\[
z_0+h=x_0+i(y_0+k),
\]
which corresponds to the point  $(x_0,y_0+k) \in \R^2$.

Thus
\begin{align*}
f'(z_0) & = \rlim{h \to 0}{h \in i\R \backslash \set{0}} \frac{f(z_0+h)-f(z_0)}{h} \\
& = \rlim{k \to 0}{k \in \R \backslash \set{0}} \frac{\left[ u(x_0,y_0+k)+iv(x_0,y_0+k) \right] - \left[ u(x_0,y_0)+iv(x_0,y_0) \right]}{ik} \\
& \vspace*{2cm} \\
& =   \rlim{k \to 0}{k \in \R \backslash \set{0}} \frac{i\ \left[ v(x_0,y_0+k)-v(x_0,y_0) \right]}{ik} +
 \rlim{k \to 0}{k \in \R \backslash \set{0}} \frac{u(x_0,y_0+k)-u(x_0,y_0)}{ik}     \\[5ex]
 & =   \rlim{k \to 0}{k \in \R \backslash \set{0}} \frac{v(x_0,y_0+k)-v(x_0,y_0)}{k} 
 -i \left( \rlim{k \to 0}{k \in \R \backslash \set{0}} \frac{u(x_0,y_0+k)-u(x_0,y_0)}{k} \right)     \\
 & = \pd{v}{y} (x_0,y_0)-i \pd{u}{y} (x_0,y_0).
\end{align*}
So $\pd{u}{y}$ and $\pd{v}{y}$ also exist at $(x_0,y_0)$, and the derivative of $f$ at $z_0$ also satisfies
\begin{equation}
\label{eq:cr2}
\tag{$\ddagger$}
f'(z_0) = f'(x_0+iy_0) = \pd{v}{y} (x_0,y_0) - i \pd{u}{y} (x_0,y_0).
\end{equation}
Equating the real and imaginary parts of the two expressions~\eqref{eq:cr1} and~\eqref{eq:cr2} for $f'(z_0)$ gives
\[
\pd{u}{x} (x_0,y_0) = \pd{v}{y} (x_0,y_0) \text{ and } \pd{v}{x}(x_0,y_0) = - \pd {u}{y} (x_0,y_0),
\]
which completes the proof.
%\vspace*{15cm}
\end{proof}
The Cauchy Riemann Equations provide a very useful way of showing that a function is \emph{not} holomorphic.  We cannot use Theorem~\ref{t:cr1} to show that a function is holomorphic.
\begin{example}
For the function $f: \C \to \C$ defined by
\[
f(z) = 
\begin{cases}
\exp(-z^{-4}) & z \neq 0 \\
0 & z=0
\end{cases}
\]
the partial derivatives of $u$ and $v$ satisfy the Cauchy-Riemann equations everywhere, but $f$ is not differentiable (nor even continuous) at $z=0$.  (This is difficult to prove).
\end{example}
\begin{comment}
\begin{remark}
A partial converse does exist: if $f:U \to \C$ is given by
\[
f(x+iy)=u(x,y)+iv(x,y)
\]
and if at all points of $U$, the partial derivatives of $u$ and $v$ (a) exist (b) satisfy the Cauchy Riemann equations and (c) are continuous, then $f$ is holomophic on $U$.  This is far more difficult to prove (and you are forbidden from using it in this module!).
\end{remark}
\begin{remark} (Note: completely irrelevant for this module)
Let $f:U \to \C$ be given and let $\mathbf{f}:U \to \R^2$ denote the corresponding real function.  If $f$ satisfies the Cauchy-Riemann equations, then $\mathbf{f}$ satisfies the \emph{Laplace Equation}
\[
\pd{{}^2\mathbf{f}}{x^2} + \pd{{}^2\mathbf{f}}{y^2}=0,
\]
and important partial differential equation that shows up in many branches of mathematics and physics.  Thus the study of holomorphic functions is closely related to the study of the Laplace equation.  Indeed, solutions to this PDE are known as \emph{harmonic functions}, and there is an entire branch of mathematics dedicated to their study.
\end{remark}
\end{comment}

\section{Geometry of Derivatives for Complex-Valued Functions}
When working with differentiable functions in $\R$, it is useful to have the geometric picture of the derivative of a function $g: \R \to \R$ - for example, by considering $g'(x)$ as the slope of the tangent to the graph of $g$ at the point $(x,g(x))$.  In this section we will try to give a geometric description of the derivative of a holomorphic function $f:U \to \C$ at a point $z_0 \in U$.  Since we cannot draw the graph of such a function, some care is needed.

Returning to the real case, consider the following graph of a differentiable function $g:\R \to \R$:
\begin{center}
\begin{tabular}{ccc}
\altgraphics[scale=0.5]{ch2_realderivative1_full}{ch2_realderivative1} & \qquad & \altgraphics[scale=0.5]{ch2_realderivative2_full}{ch2_realderivative2}
\end{tabular}
\end{center}

For sufficiently small $h$ we have
\[
\frac{g(a+h)-g(a)}{h} \approx g'(a).
\]
\begin{blankbox}
We can rewrite this approximation as
\[
g(a+h)-g(a) \approx g'(a) h.
\]
Now, at least when $g'(a) \neq 0$, we know that $g$ maps the interval $[a,a+h]$, of length $h$, to the interval $[g(a),g(a+h)]$ of length $g(a+h)-g(a)$  i.e. length approximately $g'(a)h$. 

In other words, $g$ moves $[a,a+h]$ to an interval from $g(a)$, and (approximately) scales it by a factor of $g'(a)$.
%\vspace*{3cm}
\end{blankbox}
If $g'(a)$ is negative, then $[a,a+h]$ is approximately sent to $[g(a+h),g(a)]$. 
\begin{blankbox}
Then the mapping reverses the direction of the interval, i.e., sends $[a,a+h]$ to $[g(a+h),g(a)]$.  Again, the interval is scaled by a factor of $\abs{g'(a)}$, but this time, also rotated by an angle of $\pi$.  We can represent these mappings using one-dimensional figures as shown.

{
\centering
%\altgraphics[scale=0.4]{ch2_4lines_full}{ch2_4lines}
\hidegraphics[width=0.8\textwidth]{ch2_4lines}
\showgraphics[width=\textwidth]{ch2_4lines_full}
}
\end{blankbox}

For a complex function $f:U \to \C$, its graph is the set of points
\[
\set{ (z,f(z)): z \in U },
\]
as subset of $\C^2$.  We would need 4 coordinates to draw such a graph, which is impossible.

\begin{question}
How do we describe the geometry of derivatives of complex functions?
\end{question}

\begin{answer}
To describe the derivative of $f'(z_0)$ for a point $z_0 \in U$ geometrically, we will look at an analogy of the real case we have described above.   Indeed, rather than looking at a small interval of the form $[a,a+h]$, we look at a small disk $D(z_0,r)$ centred at  $z_0$.  Again, let us restrict to the case where $g'(z_0) \neq 0$.

If $r$ is `small' and $z=z_0+h \in D(z_0,r)$ then again
\[
\frac{f(z_0+h)-f(z_0)}{h} \approx f'(z_0).
\]

This can be rewritten as
\[
f(z_0+h) - f(z_0)\approx hf'(z_0), \quad\text{i.e.}\quad f(z)-f(z_0) \approx hf'(z_0).
\]
If we think of $h$ as the vector from $z_0$ to $z$, and $f(z)-f(z_0)$ as the vector from $f(z_0)$ to $f(z)$, then roughly speaking, $f$  sends $h$ to $hf'(z_0)$.  In other words, the vector $h$ gets moved to a vector from $f(z_0)$, and approximately gets multiplied by $f'(z_0)$, i.e.
\begin{itemize}
\item scaled by a factor of $\abs{f'(z_0)}$ and
\item rotated by angle $\arg (f'(z_0))$ anticlockwise about $f(z_0)$.
\end{itemize}
Again, we are using the fact that for $z_1,z_2 \in \C$, $\abs{z_1z_2} = \abs{z_1} \abs{z_2}$ and $\arg (z_1z_2) = \arg(z_1)+\arg(z_2)$.

\begin{center}
\begin{tabular}{cc}
\hidegraphics[scale=1]{ch2_deriv3} & \qquad \hidegraphics[scale=1]{ch2_deriv4} \\
\end{tabular}
\showgraphics[width=\textwidth]{ch2_complexderiv_full}
\end{center}
\end{answer}
Since the above remarks hold for any $z \in D(z_0,r)$, the mapping $f$ approximately transforms all points in $D(z_0,r)$ in the same way.
\begin{summary}
Let $f$ be differentiable at a point $z_0 \in \C$ with $f'(z_0) \neq 0$, then $f$ approximately maps small disks centred at $z_0$ to small disks centred at $f(z_0)$ as follows:
\begin{center}
\emph{scaling by a factor of $\abs{f'(z_0)}$ and rotating by an angle of $\arg (f'(z_0))$ anticlockwise about $f(z_0)$.}
\end{center}
\end{summary}
\begin{figure}[H]
\centering
\altgraphics[width=\textwidth]{ch2_geometry_deriv_full}{ch2_geometry_deriv}
\caption{The image under $f$ of a small disc centred at $z_0$.}
\end{figure}

\begin{example}
Find the geometric effect of applying the function
\[
f(z)=z^2-\frac{i}{z^2}
\]
to a small disk centred at $i$.
\end{example}
%\vspace*{12cm}
\begin{solution}
Since $f$ is differentiable at $i$, the previous remarks indicate that a small disc centred at $i$ gets sent to a small disc centred at $f(i)$, where
\[
f(i) = i^2-\frac{i}{i^2} = -i+1.
\]
To determine the geometric effect of $f$ applied to this disc, we need to calculate $\abs{f'(i)}$ and $\arg (f'(i))$.  The rules of differentiation tell us that
\[
f'(z) = 2z-i \left( \frac{-2}{z^3} \right) = 2z+ \frac{2i}{z^3},
\]
and hence $f'(i) = 2i-2$, which has modulus $\abs{2i-2}=2\sqrt{2}$ and argument $3\pi/4$.

Hence a small disc at $i$ is approximately mapped to a small disc at $-1+i$, and is scaled by a factor of $2\sqrt{2}$ and rotated by an angle of $3\pi/4$  in the anticlockwise direction about $-1+i$.

\end{solution}

\begin{figure}[H]
\centering
\hidegraphics[scale=1]{ch2_parametric3} \quad \hidegraphics[scale=1]{ch2_parametric4}
\showgraphics[scale=0.5]{ch2_parametric3_full}
\caption{The geometric effect of applying $f(z)=z^2-\dfrac{i}{z^2}$ to small circles centred at $i$. The transformed circles are centred at $f(i)$.  As Figure~\ref{f:circles} shows, the images of the smaller circles are almost circular, but the larger ones less so.  The horizontal line from $i$ is rotated by $\arg(f'(i))$.}
\label{f:circles}
\end{figure}

\begin{example}
Determine the geometric effect of applying the function $f$, where $f(z)=z^3$, to a small disc centred at $z_0 = \sqrt{3}-i$.
\end{example}
\begin{solution}
This time, the disc gets sent to a disc centred at $f(z_0) = (\sqrt{3}-i)^3 = -4i$.  Since $f'(z)=3z^2$, we have
\[
f'(\sqrt{3}-i) = 3 ( \sqrt{3}-i)^2 = 3 (2-i2\sqrt{3}) = 6-i6\sqrt{3}.
\]
Thus
\[
\abs{f'(\sqrt{3}-i) } = \sqrt{(6)^2+\left(6\sqrt{3}\right)^2 } = 12,
\]
and
\[
\arg \left( f' (\sqrt{3}-i ) \right) = \arg \left( 6-i6\sqrt{3} \right) = - \pi/3.
\]

Hence a small disc at $\sqrt{3}-i$ is approximately mapped to a small disc at $-4i$, and is scaled by a factor of $12$ and rotated by an angle of $-\pi/3$  in the \emph{clockwise} direction about $-4i$ (note: this time the rotation is clockwise since $\arg (f'(z_0))$ is negative).
\end{solution}




