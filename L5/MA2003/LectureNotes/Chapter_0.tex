
\section*{Module Information}

\subsection*{Lecturer}
David McConnell, McConnellD@Cardiff.ac.uk, Office M/0.25, Ground Floor, Mathematics Building

\subsection*{Lecture Notes}
Printed outline lecture notes will be handed out during lectures, in several parts.  There are six chapters in total.  These will contain definitions, statements of main results etc., with gaps to be filled in during class (with examples, comments and so on).  I have not left blank spaces for proofs (except in one or two cases), as many of them are quite long.

To get the most out of the lectures in this module, you are advised to read ahead in the notes and attempt to `fill in the blanks' before coming to lectures.

Complete lecture notes will be posted online, after each section has been completed.

\subsection*{Texbook}

The recommended textbook for this module is
\begin{itemize}
\item \emph{Basic Complex Analysis}, J.E. Marsden and M.J. Hoffman, (Freeman).
\end{itemize}
It is not necessary to buy this textbook; there are numerous copies available in the library.  There are a number of excellent alternatives, such as
\begin{itemize}
\item \emph{Complex Analysis}, J.M. Howie, (Springer),
\item \emph{Introduction to Complex Analysis}, H.A. Priestley (Oxford),
\item \emph{Complex Analysis}, I. Stewart and D. Tall (Cambridge),
\item \emph{Lecture Notes on Complex Analysis}, I.F. Wilde (Imperial College Press),
\end{itemize}
and many others.


\subsection*{Office Hours}
\begin{itemize}
\item Monday: 15.00-16.00,
\item Tuesday: 16.00-17.00.
\end{itemize}
I am happy to answer any questions you might have about the material covered in lectures or on the problem sheets during office hours.  You do not have to let me know in advance if you plan on coming at these times.\\~\\
Outside of these times, it is probably best to email first to arrange a convenient time to meet.



 


\subsection*{Exercise Sheets}

There will be 6 problem sheets over the course of this module. The first of these consists of revision exercises, on topics that have seen in Year 1.  You are advised to attempt these as soon as possible.  

I will discuss the content of the problem sheets during the one of Wednesday lectures each fortnight.  I will let you know in advance which problems I intend to cover during these classes.

You do not have to hand in you attempts at the problem sheets, though you are welcome to do so if you would like me to mark them, and to provide feedback.  Outline solutions will be posted on Learning Central.


\subsection*{Assessment}
This module will be assessed with a 2 hour exam.  

\subsection*{Discussion Forum}
I have created a discussion forum for this module, to access it please follow the link on Learning Central.  You can post any questions about the course content here, and anyone can answer them (including me).  As a general rule, this is preferable to asking questions by email, particularly mathematical questions, as the discussion board supports \LaTeX.  

\subsection*{Course Content}
\begin{itemize}
\item Revision of complex numbers, functions of a complex variable.
\item Continuity, differentiability and holomorphic functions.
\item Path Integrals.  When we integrate `between' two complex numbers $z_1$ and $z_2$, we must take into account the fact that there are many different paths between $z_1$ and $z_2$.
\item Cauchy's Theorem, Cauchy's Integral Formula and the Residue Theorem.  These results concern integrals along paths that have the same start- and end-points.
\item Applications: Taylor's Theorem, Evaluating Real Integrals using Residues.  For example, we will show how many useful real integrals can be evaluated using techniques from complex analysis, e.g., integrals of the form
\[
\int_{-\infty}^{+\infty} \frac{1}{x^4+1}\ dx \quad\text{or}\quad \int_0^{2\pi} \frac{\sin(t)}{5+2\sin(t)}\ dt.
\]
Note that these cannot be evaluated using methods from calculus.
\end{itemize}
