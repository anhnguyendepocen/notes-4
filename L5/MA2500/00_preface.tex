% !TEX root = main.tex

%-------------------------------------------------
\chapter*{Preface}\label{sec:preface}
\addcontentsline{toc}{chapter}{Preface}

In \textsl{MA2500 Foundations of Probability and Statistics} we build a theory of probability and statistics from first principles and as such the module can be studied independently of the first year modules \textsl{MA1500 Introduction to Probability Theory} and \textsl{MA1501 Statistical Inference}. The module however does assume a good understanding of many fundamental mathematical ideas and techniques covered in \textsl{MA1005 Foundations of Mathematics I} and \textsl{MA1006 Foundations of Mathematics II}.

%-------------------------------------------------
\subsection*{Timetable}
\begin{tabular}{llll}
Monday			& 16.10 -- 17.00	& Marking session			& E/0.15 \\
Tuesday			& 15.10 -- 17.00	& Lectures and discussion	& E/0.15 \\ 
Thursday		& 12.10 -- 14.00	& Lectures and discussion	& E/0.15 \\ 
\end{tabular}

%-------------------------------------------------
\subsection*{Topics}
\begin{tabular}{llcll} 
Week 1		& Elementary probability	&\mbox{}\quad\qquad{} & Week 7		& Estimation \\
Week 2		& Probability spaces 		& & Week 8		& Likelihood \\
Week 3		& Random variables 			& & Week 9		& Hypothesis testing \\
Week 4		& Expectation 				& & Week 10		& Non-parametric methods \\
Week 5		& Sums of random variables 	& & Week 11		& Bivariate analysis \\
Week 6		& Joint distributions 		& & Week 12		& Random processes \\ 
\end{tabular}

%-------------------------------------------------
\subsection*{Instructions}
\begin{tabular}{l} 
Read handout notes before lectures. \\
Attend all lectures and marking sessions. \\
Ask questions during lectures and marking sessions. \\
Work with peers on exercises and participate in marking sessions. \\
Ask for help when required. \\ 
\end{tabular}

%-------------------------------------------------
\subsection*{Handouts}
These handout notes contain only definitions, statements of theorems, exercise questions and partial examples. 
During lectures you should annotate these notes and use a separate notebook for proofs, sketches and longer examples.

\bigskip
\fbox{\fbox{
\begin{minipage}{\textwidth}
\centering
\textbf{You are expected to read the relevant parts of the handout notes \emph{before} lectures.}
\end{minipage}}}
%-------------------------------------------------
\subsection*{Exercises}
Learning mathematics takes exercise, which invariably involves a lot of hard work. When solutions to exercises are available we can be tempted to give up the struggle and look at the solutions. Struggling with exercises however is by far the best way to \emph{understand mathematics}, and is the only way we can learn how to \emph{do mathematics}. Nobody expects to get fit by watching someone working on an exercise bike: we need to get on the bike ourselves and start pedalling, which of course involves a lot of hard work. For this reason, \emph{written solutions to exercises will not be provided} as a matter of course, but individual solutions will be provided on request, perhaps as short videos. Answers to exercises can be submitted at any time for assessment and feedback. This extends to any exercise contained in either of the recommended textbooks for the module. Every exercise is a challenge.

%-------------------------------------------------
\subsection*{Marking sessions}
The ability to criticise your own work is very important, but it does not always come easily. To develop our critical faculties we can first learn to criticise other people's work, before applying the same principles to our own. For this reason we will follow a weekly cycle of \emph{peer marking}. Every week teams of 2--4 students will prepare written answers to selected exercise questions and bring these along to peer marking sessions. During these sessions, solutions will be presented on the board while and teams mark and provide feedback on each others' work using a standard marking form. The marked scripts will be collected at the end of each session and returned the following week with additional feedback included where necessary. 

\smallskip
All answers should aim to meet the criteria shown in Table~\ref{tab:assessment_criteria} and marked accordingly.
\begin{table}[htb]
\centering
\begin{tabular}{|lp{0.5\linewidth}|}\hline
\textbf{Clear}		& Statements are explicit and unambiguous. \\
\textbf{Complete}	& All relevant details are included. \\
\textbf{Concise}	& No irrelevant details are included. \\
\textbf{Coherent}	& An appropriate narrative is provided. \\
\textbf{Correct}	& Arguments are precise and logically sound. \\ \hline
\end{tabular}
\caption{Assessment criteria for homework exercises.\label{tab:assessment_criteria}}
\end{table}

%-------------------------------------------------
\subsection*{Recommended textbooks}
\ben
\it
Probability and Random Processes (Third edition) \\
G. R. Grimmett and D. R. Stirzaker \\
Oxford University Press (2001) \\
ISBN 0-19-857222-0 \\[2ex]
\it	
Introduction to Mathematical Statistics (Sixth edition) \\
R. V. Hogg, J. W. McKean and A. T. Craig \\
Prentice Hall (2005) \\
ISBN 0-13-122605-3 \\
\een

%-------------------------------------------------
\subsection*{A brief history of probability}\label{sec:history}

In the 17th century, gambling was popular among the French aristocracy. A popular game involved rolling a pair of fair dice 24 times and betting even money on a ``double six'' appearing at least once. It was understood that betting on at least one six in four rolls of a single fair die was profitable, and it was believed that the same reasoning extended to betting on at least one double six in 24 rolls. A French nobleman, Chevalier de M\'{e}r\'{e}, knew from experience that when betting on a six appearing in four rolls of a single die he won more times than he lost, but when betting on a double six appearing in 24 rolls of two dice, he lost more times than he won.
\begin{itemize}
\item
In 1654 de M\'{e}r\'{e} asked his friend, the mathematician Blaise Pascal (1623-1662), to explain this apparent paradox. The problem initiated an exchange of letters between Pascal and Pierre de Fermat (1601-1665), which led to the formulation of the classical principles of probability theory.
\item
In 1657 the Dutch scientist Christiaan Huygens (1629-1695) published the first book on probability, \textit{De Ratiociniis in Ludo Aleae} (\textit{On Reasoning in Games of Chance}) which introduced the concept of mathematical expectation.
\item
In 1713 the celebrated book \textit{Ars Conjectandi} by Swiss mathematician Jakob Bernoulli (1654-1705) was published. It contains a theorem known as the law of large numbers, the first limit theorem in probability theory. 
\item 
In 1718 Abraham de Moivre (1667-1754) published \textit{The Doctrine of Chances}. This was the first textbook on probability theory and is said to have been prized by gamblers.
\item 
In 1812 Pierre de Laplace (1749-1827) introduced many new ideas and techniques in his book, \textit{Th\'{e}orie Analytique des Probabilit\'{e}s}. Laplace applied probabilistic ideas to many scientific and practical problems as well as to games of chance. 
\item 
In 1919 Richard von Mises (1883-1953) established an axiomatic approach to the subject based on relative frequency, and introduced the idea of sample spaces.
\item
In 1933 Andrey Kolmogorov (1903-1987) introduced the modern axiomatic theory of probability, which is part of a more general field known as \emph{measure theory}.
\end{itemize}

Our study will be based on Kolmogorov's axiomatic theory.
