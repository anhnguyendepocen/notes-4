% !TEX root = main.tex

%-------------------------------------------------
\section{Appendix}\label{sec:}

\subsection{Proof of the Cauchy-Schwarz inequality for random variables}

We need to show that for any two random variables $X$ and $Y$,
\[
\expe(XY)^2 \leq \expe(X^2)\expe(Y^2)
\]
with equality if and only if $\prob(Y=aX)=1$ for some $a\in\R$.

First we need the following technical result (which we shall not prove).
% lemma
\begin{lemma}\label{lem:pos_rv_expe_zero}
If $X\geq 0$ and $\expe(X)=0$ then $\prob(X=0)=1$.
\end{lemma}

\begin{proof}[Cauchy-Schwarz]
$X^2$ and $Y^2$ are non-negative random variables, so by Lemma~\ref{lem:pos_rv_expe_zero} we can assume that $\expe(X^2)>0$ and $\expe(Y^2)>0$ (otherwise both sides of the inequality are identically zero).

Let $a\in\R$ consider the random variable $Z=aX-Y$. By the properties of expectation,
\begin{align*}
Z^2\geq 0 
	& \Rightarrow \expe(Z^2)\geq 0  \qquad\text{(positivity)}\\
	& \Rightarrow \expe(a^2X^2 - 2aXY + Y^2) \geq 0  \\
	& \Rightarrow a^2\expe(X^2) - 2a\expe(XY) + \expe(Y^2) \geq 0 \qquad\text{(linearity).}
\end{align*}

\bit
\it Let $q(a)=a^2\expe(X^2) - 2a\expe(XY) + \expe(Y^2)$. This is a quadratic expression in $a$.
\eit
Since $q(a)\geq 0$ for all $a\in\R$, the roots of the quadratic equation $q(a)=0$, given by
\[
a = \frac{\expe(XY)\pm\sqrt{\expe(XY)^2-\expe(X^2)\expe(Y^2)}}{\expe(X^2)}
\]
are either both complex (discriminant is negative) or co-incide (discriminant is zero). 
\bit
\it Hence $\expe(XY)^2-\expe(X^2)\expe(Y^2) \leq 0$, or equivalently $\expe(XY)^2\leq \expe(X^2)\expe(Y^2)$.
\eit

The discriminant is zero if and only if the quadratic has one real root, which occurs if and only if $\expe(Z^2)=0$ for some $a\in\R$, i.e.
\[
\expe\left((aX-Y)^2\right) = 0\qquad\text{for some $a\in\R$.}
\]
Because $(aX-Y)^2$ is a non-negative random variable, we have by Lemma~\ref{lem:pos_rv_expe_zero} that
\[
\prob\big[(aX-Y)^2 = 0\big] = 1
\]
or equivalently, $\prob\big(Y = aX) = 1$, as required. 
\end{proof}
