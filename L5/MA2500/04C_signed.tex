% !TEX root = main.tex

%-------------------------------------------------
\section{Signed variables}\label{sec:}

Random variables that take both positive and negative values are called \emph{signed variables}. 

\begin{definition}
The \emph{positive part} and \emph{negative part} of a random variable $X$ are
\begin{align*}
X^{+}(\omega) 
	& = \begin{cases} 
			X(\omega)	& \text{if } X(\omega)\geq 0, \\
	   		0 		 	& \text{if } X(\omega) < 0, \text{ and}
	\end{cases} \\[2ex]
X^{-}(\omega)	
	& = \begin{cases} 
			-X(\omega)	& \text{if } X(\omega) < 0, \\
			0 		 	& \text{if } X(\omega) \geq 0.
		\end{cases}
\end{align*}
respectively. 
\end{definition}

Note that $X^{+}$ and $X^{-}$ are both non-negative random variables with
\[
X = X^{+} - X^{-}
\qquad\text{and}\qquad
|X| = X^{+} + X^{-}.
\]

\begin{definition}[Expectation of signed random variables]
The \emph{expectation} of a signed random variable $X$ is
\[
\expe(X) = \expe(X^{+}) - \expe(X^{-})
\]
provided that $\expe(X^{+})$ and $\expe(X^{-})$ are not both infinite.
\end{definition}

\begin{remark}
If $\expe(X^{+})$ and $\expe(X^{-})$ are both infinite, the expectation of $X$ is \emph{undefined} because we cannot make sense of the expression ``$\infty - \infty$''. In this case we say that the expectation of $X$ \emph{does not exist}.
\end{remark}

\begin{example} % discrete, undefined
Let $X$ be a discrete random variable with the following PMF:
\[
f(k) = \begin{cases}
	\displaystyle\frac{3}{\pi^2 k^2} 	& \text{if }\ k\in\{\pm 1,\pm 2, \ldots\} \\[1ex]
	0						& \text{otherwise.}
\end{cases}	
\]
Show that $\expe(X)$ is undefined.
\begin{solution}
$X$ is a signed random variable, so we must deal with its positive and negative parts separately:
\[
X^{+}	= \begin{cases}  X & \text{ if } X \geq 0, \\ 0 & \text{otherwise.}\end{cases}
\qquad\text{and}\qquad
X^{-}	= \begin{cases} -X & \text{ if } X <    0, \\ 0 & \text{otherwise.}\end{cases}
\]
The expected values of $X^{+}$ and $X^{-}$ are
\[\begin{array}{lll}
\expe(X^{+})	
	& = \displaystyle \sum_{k=1}^{\infty} k \left(\frac{3}{\pi^2 k^2}\right)
	& = \displaystyle \frac{3}{\pi^2} \sum_{k=1}^{\infty}\frac{1}{k} = \infty \\[2ex]
\expe(X^{-})	
	& = \displaystyle \sum_{k=-\infty}^{-1} (-k) \left(\frac{3}{\pi^2 k^2}\right)
	& = \displaystyle \frac{3}{\pi^2} \sum_{k=1}^{\infty}\frac{1}{k} = \infty \\
\end{array}\]
so $\expe(X) = \expe(X^{+}) - \expe(X^{-})$ is undefined.
\end{solution}
\end{example}

\begin{example}[Cauchy distribution]% continuous - undefined
Let $X$ be a continuous random variable having the following PDF,
\[
f(x) = \frac{1}{\pi(1+x^2)}\qquad\text{for all $x\in\R$.}
\]
Show that $\expe(X)$ is undefined.
\begin{solution}
$X$ is a signed variable, so we must deal with its positive and negative parts separately:

\[\begin{array}{lll}
\expe(X^{+}) 
	& = \displaystyle\int_0^\infty xf(x)\,dx 
	& = \displaystyle\frac{1}{\pi}\int_0^\infty \frac{x}{1+x^2}\,dx  \\[2ex]
\expe(X^{-})	
	& = \displaystyle\int_{-\infty}^0 (-x)f(x)\,dx
	& = \displaystyle\frac{1}{\pi}\int_0^{\infty} \frac{x}{1+x^2}\,dx
\end{array}\]
Thus $\expe(X) = \expe(X^{+}) - \expe(X^{-})$ is given by
\begin{align*}
\expe(X)
%	& = \expe(X^{+}) - \expe(X^{-}) \\
	& = \frac{1}{\pi}\left[\int_0^{\infty} \frac{x}{1+x^2}\,dx - \int_0^{\infty} \frac{x}{1+x^2}\,dx\right].
\end{align*}

At this point, it might be tempting to conclude that $\expe(X)=0$. However if $x>1$ then $x^2 > 1$ and therefore $2x^2 > 1+x^2$, so
\[
\frac{x}{1+x^2} > \frac{1}{2x} \qquad\text{for all } x > 1
\]
Consequently,
\[
\int_{0}^{\infty}\frac{x}{1+x^2}\,dx
	> \int_{1}^{\infty}\frac{x}{1+x^2}\,dx
	> \frac{1}{2}\int_{1}^{\infty}\frac{1}{x}\,dx
	= \infty,
\]
so $\expe(X)$ is undefined.
\end{solution}
\end{example}

