% !TEX root = main.tex

%-------------------------------------------------
\section{Conditional probability}\label{sec:cond_prob}

Recall from elementary probability that the conditional probability of $A$ given $B$ is defined by
\[
\prob(A|B) = \frac{\prob(A\cap B)}{\prob(B)}.
\]
provided that $P(B)>0$. The following theorem shows that this defines a probability measure on subsets of $\Omega$.

\begin{theorem}
Let $\prob$ be a probability measure on subsets of $\Omega$ and let $B$ be an event with $\prob(B)>0$. Then the function
\[
\mathbb{Q}(A) = \prob(A|B)
\]
is also probability measure on subsets of $\Omega$.

\begin{proof}
To show that $\mathbb{Q}$ is a probability measure, we need to verify that $\mathbb{Q}(\emptyset) = 1$, $\mathbb{Q}(\Omega) = 1$ and that $\mathbb{Q}$ is countably additive. Firstly,
\begin{align*}
\mathbb{Q}(\emptyset) 
	& = \prob(\emptyset|B) = \prob(\emptyset\cap B)/\prob(B) = \prob(\emptyset)/\prob(B) = 0; \\
\mathbb{Q}(\Omega) 
	& = \prob(\Omega|B) = \prob(\Omega\cap B)/\prob(B) = \prob(B)/\prob(B) = 1.
\end{align*}
To prove that $\mathbb{Q}$ is countable additive, let $A_1,A_2,\ldots$ be pairwise disjoint subsets of $\Omega$. Then using the fact that $\prob$ is countably additive,
\begin{align*}
\mathbb{Q}(\medcup_{i=1}^{\infty} A_i) 
	= \prob(\medcup_{i=1}^{\infty} A_i\,|\,B)
	& = \frac{\prob\big[(\medcup_{i=1}^{\infty} A_i) \cap B\big]}{\prob(B)} \\
	& = \frac{\prob\big[\medcup_{i=1}^{\infty} (A_i \cap B)\big]}{\prob(B)} \quad\text{(because intersection is distributive over union)},\\
	& = \frac{\sum_{i=1}^{\infty} \prob(A_i \cap B)}{\prob(B)} \quad\text{(because the sets $A_i\cap B$ are disjoint)}, \\
	& = \sum_{i=1}^{\infty}\frac{\prob(A_i \cap B)}{\prob(B)}
	  = \sum_{i=1}^{\infty}\prob(A_i|B)
		= \sum_{i=1}^{\infty} \mathbb{Q}(A_i)\quad\text{as required.}
\end{align*}
\end{proof}
\end{theorem}

