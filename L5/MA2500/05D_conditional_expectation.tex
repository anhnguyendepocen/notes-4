% !TEX root = main.tex

%-------------------------------------------------
\section{Conditional expectation}\label{sec:cond_expe}

Conditional expectations are expectations computed with respect to conditional distributions.
%PMFs (discrete case) and conditional PDFs (continuous case).

%%-----------------------------
%\subsection{Conditioning on events}
%\begin{definition}\label{defn:cond_expe_A}
%let $A$ be an event with $\prob(A)>0$, and let $X$ be a random variable. Then
%\[
%\begin{array}{lll}
%\expe(X|A) & = \displaystyle\sum_{i=1}^{\infty} x_i\,f_{X|A}(x_i)		& \quad\text{if $X$ is discrete, or} \\[4ex]
%\expe(X|A) & = \displaystyle\int_{-\infty}^{\infty} x\,f_{X|A}(x|y)\,dy	& \quad\text{if $X$ is continuous.} 
%\end{array}
%\]
%\end{definition}
%
%% example: cond_expe for cts. uniform
%\begin{example}
%Let $X\sim\text{Uniform}[0,1]$ and let $A=\{1/2\leq X\leq 3/4\}$. The CDF of $X$ is $F_X(x) = x$ for $0\leq x\leq 1$ (with $F_X(x)=0$ for $x<0$ and $F_X(x)=1$ for $x>1$) so
%\[
%\prob(A) = \prob(1/2\leq x\leq 3/4) = \prob(X\leq 3/4)-\prob(X\leq 1/2) = F_X(3/4) - F_X(1/2) = 3/4 - 1/2 = 1/4.
%\]
%The PDF of $X$ is $f_X(x)=1$ for $0\leq x\leq 1$ (and zero otherwise), so the conditional PDF of $X|A$ is
%\[
%f_{X|A}(x) = \frac{f_X(x)}{\prob(A)} = 4 \text{for $1/2\leq x\leq 3/4$ (and zero otherwise),}
%\]
%which shows that $X|A\sim\text{Uniform}[1/2,3/4]$. The conditional expectation of $X|A$ is
%\[
%\expe(X|A) 
%	= \int_{-\infty}{\infty} x f_{X|A}(x)\,dx
%	= \int_{1/2}{3/4} 4x \,dx
%	= 5/8.
%\]
%This is the mid-point of the interval $[1/2,3/4]$ as expected.
%\end{example}

%\begin{example}
%Recall example~\label{example:cond_pmf}: a fair coin is tossed repeatedly until a head occurs; $X$ is the number of times the coin is tossed; $A$ is the event that the coin is tossed an odd number of times. If $A$ occurs, the expcted value of $X$ is
%\[
%\expe(X|A) = \sum_{k\text{odd}} k(3/2^{k+1}) = \sum_{k\text{odd}}^{\infty} 3/2^k  = 3(1/2 + 1/8 + 1/32 + \ldots) = 2.
%\]
%The unconditional expectation is 
%\[
%\expe(X) = \sum_{k=1}^{\infty} k(1/2^k) = \sum_{k=1}^{\infty} 1/2^{k-1} = 1 + 1/2 + 1/4 + 1/8 + \ldots = 2.
%\]
%*** so it doesn't change!? ***
%\end{example}

\begin{definition}\label{def:cond_expe_x}
Let $x$ be a fixed value. The \emph{conditional expectation of $Y$ given $X=x$} is 
\[
\begin{array}{lll}
\expe(Y|X=x) & = \displaystyle\sum_{j=1}^{\infty} y_j\,f_{Y|X}(y_j|x)		& \quad\text{if $Y$ is discrete, or} \\[4ex]
\expe(Y|X=x) & = \displaystyle\int_{-\infty}^{\infty} y\,f_{Y|X}(y|x)\,dy	& \quad\text{if $Y$ is continuous.} 
\end{array}
\]
\end{definition}

%-----------------------------
%\subsection{Conditioning on events}
%\begin{definition}\label{defn:cond_expe_A}
%let $A$ be an event with $\prob(A)>0$, and let $X$ be a random variable. Then
%\[
%\begin{array}{lll}
%\expe(X|A) & = \displaystyle\sum_{i=1}^{\infty} x_i\,f_{X|A}(x_i)		& \quad\text{if $X$ is discrete, or} \\[4ex]
%\expe(X|A) & = \displaystyle\int_{-\infty}^{\infty} x\,f_{X|A}(x|y)\,dy	& \quad\text{if $X$ is continuous.} 
%\end{array}
%\]
%\end{definition}

\begin{definition}
Let $A$ be an event with $\prob(A)>0$. The \emph{conditional expectation of $Y|A$} is the conditional expectation of $Y$ given $I_A=1$, where $I_A$ is the indicator variable of $A$.
%%\[
%%\begin{array}{cccl}
%%F_{X|A}:	& \R	& \to		& [0,1] \\
%%			& x		& \mapsto	& \prob(X\leq x\,|\, A).
%%\end{array}
%%\]
%The conditional PMF/PDF of $X|A$ are the functions
%\[
%\begin{array}{lll}
%f_{X|A}(x) & = \prob(\{X=x\}\cap A)/\prob(A)& \quad\text{if $X$ is discrete, or} \\
%f_{X|A}(x) & = f_{X}(x)/\prob(A)			& \quad\text{if $X$ is continuous.} 
%\end{array}
%\]
\end{definition}


% example: cond_expe for cts. uniform
\begin{example}
Let $Y\sim\text{Uniform}[0,1]$. Find the conditional expectation of $Y$ given that $1/2\leq Y\leq 3/4$.

\begin{solution}
Let $A=\{1/2\leq X\leq 3/4\}$. 

The CDF of $Y$ is $F_Y(y) = y$ for $0\leq y\leq 1$ (with $F_Y(y)=0$ for $y<0$ and $F_Y(y)=1$ for $y>1$) so
\[
\prob(A) = \prob(1/2\leq Y\leq 3/4) = \prob(Y\leq 3/4)-\prob(Y\leq 1/2) = F_Y(3/4) - F_Y(1/2) = 3/4 - 1/2 = 1/4.
\]

The PDF of $Y$ is $f_Y(y)=1$ for $0\leq y\leq 1$ (and zero otherwise), so the conditional PDF of $Y|A$ is
\[
f_{Y|A}(y) = \frac{f_Y(y)}{\prob(A)} = 4 \quad\text{for $1/2\leq x\leq 3/4$ (and zero otherwise),}
\]
which shows that $Y|A\sim\text{Uniform}[1/2,3/4]$. The conditional expectation of $Y|A$ is
\[
\expe(Y|A) 
	= \int_{-\infty}^{\infty} y f_{Y|A}(y)\,dy
	= \int_{1/2}^{3/4} 4y \,dy
	= 5/8.
\]
This is the mid-point of the interval $[1/2,3/4]$ as expected.
\end{solution}
\end{example}

%-----------------------------
\subsection{Conditioning on random variables}

For any fixed value of $x$, the conditional expectation $\expe(Y|X=x)$ is just a number. Let us now think of $x$ as a variable quantity, and consider the transformation
\[
\begin{array}{rccl}
	g:	& \R	& \to		& \R \\
		& x		& \mapsto	& \expe(Y|X=x)
\end{array}
\]
This transformation of $X$ yields a new random variable called the \emph{conditional expectation of $Y$ given $X$}.
\begin{definition}\label{def:cond_expe}
The \emph{conditional expectation of $Y|X$} is the random variable
\[\begin{array}{llll}
\expe(Y|X):	& \Omega 	& \to 		& \R \\
			& \omega	& \mapsto 	& \expe\big[Y|X=X(\omega)\big].
\end{array}\]
\end{definition}

The distribution of $\expe(Y|X)$ depends only on the distribution of $X$, and its expectation is given by

\[
\begin{array}{lll}
\expe\big[\expe(Y|X)\big] & = \displaystyle\sum_{i=1}^{\infty}\expe(Y|X=x_i)f_X(x_i)		& \quad\text{if $X$ is discrete, or} \\[3ex]
\expe\big[\expe(Y|X)\big] & = \displaystyle\int_{-\infty}^{\infty} \expe(Y|X=x)f_X(x)\,dx	& \quad\text{if $X$ is continuous.} 
\end{array}
\]

\begin{theorem}[Law of total expectation]
Let $X$ and $Y$ be random variables defined on the same probability space. Then
\[
\expe(Y) = \expe\big[\expe(Y|X)\big].
\]
\end{theorem}
\begin{proof}
For discrete random variables (the continuous case is similar):
\begin{align*}
\expe\big[\expe(Y|X)\big] 
	= \sum_x \expe(Y|X=x) f_X(x) 
	& = \sum_x\left(\sum_y y\,f_{Y|X}(y|x)\right) f_X(x) \\
	& = \sum_x\left(\sum_y y\,\frac{f_{X,Y}(x,y)}{f_X(x)}\right) f_X(x) \\
	& = \sum_x\left(\sum_y y f_{X,Y}(x,y)\right) \\
	& = \sum_x y \left(\sum_y f_{X,Y}(x,y)\right) \\
	& = \sum_x y f_Y(y)  
	= \expe(Y).
\end{align*}
%For continuous random variables (the discrete case is similar):
%\begin{align*}
%\expe\big[\expe(Y|X)\big] 
%	& = \int_{-\infty}^{\infty} \expe(Y|X=x) f_X(x)\,dx \\
%	& = \int_{-\infty}^{\infty}\left(\int_{-\infty}^{\infty} y\,f_{Y|X}(y|x)\,dy\right) f_X(x)\,dx \\
%	& = \int_{-\infty}^{\infty}\left(\int_{-\infty}^{\infty} y\,\frac{f_{X,Y}(x,y)}{f_X(x)}\,dy\right) f_X(x)\,dx \\
%	& = \int_{-\infty}^{\infty}\left(\int_{-\infty}^{\infty} y f_{X,Y}(x,y)\,dy\right)\,dx \\
%	& = \int_{-\infty}^{\infty} y \left(\int_{-\infty}^{\infty} f_{X,Y}(x,y)\,dx\right)\,dy \\
%	& = \int_{-\infty}^{\infty} y f_Y(y) \,dy \\
%	& = \expe(Y).
%\end{align*}
\end{proof}

\begin{theorem}[Law of total variance]
If $X$ and $Y$ are random variables defined on the same probability space,
\[
\var(Y) = \expe\big[\var(Y|X)\big] + \var\big[\expe(Y|X)\big] 
\]
where $\var(Y|X)=\expe(Y^2|X) - \expe(Y|X)^2$.
\end{theorem}
\begin{proof}
By the law of total expectation,
\begin{align*}
\var(Y)
	& = \expe(Y^2) - \expe(Y)^2 \\
	& = \expe\big[\expe(Y^2|X)\big] - \expe\big[\expe(Y|X)\big]^2
\end{align*}
Because $\var(Y|X)=\expe(Y^2|X) - \expe(Y|X)^2$,
\[
\var(Y) = \expe\big[\var(Y|X) + \expe(Y|X)^2\big] - \expe\big[\expe(Y|X)\big]^2
\]
Hence, by the linearity of expectation,
\begin{align*}
\var(Y)
	& = \expe\big[\var(Y|X)\big] + \Big(\expe\big[\expe(Y|X)^2\big] - \expe\big[\expe(Y|X)\big]^2\Big) \\
	& = \expe\big[\var(Y|X)\big] + \var[\expe(Y|X)\big].
\end{align*}
\end{proof}

% example
\begin{example}
Let the joint PDF of the continuous random variables $X$ and $Y$ be 
\[
f_{X,Y}(x,y) = \begin{cases}
	cxy & \quad\text{for $x,y\geq 0$ with $x+y\leq 1$}, \\
	0	& \quad\text{otherwise.}
\end{cases}
\]
\ben
\it Sketch the support of $f_{X,Y}$
\it Show that $c=24$.
\it Compute the conditional expectation $\expe(Y|X)$.
\it Verify the identity $\expe\big[\expe(Y|X)\big]=\expe(Y)$.
\een 

\begin{solution}
\ben

\it % << (a)
$\supp(f_{X,Y})$ is the lower-left half of the unit square.

\it % << (b)
The marginal PDF of $X$ is
\[
f_X(x) = c\int_{0}^{1-x} xy\,dy = \frac{cx(1-x)^2}{2}
\]
To find $c$,
\[
\int_{0}^{1} f_X(x)\,dx = 1, \qquad\text{so}\quad c=24.
\]
Thus
\begin{align*}
f_X(x)	& = 12x(1-x)^2 \qquad 0\leq x\leq 1 \\
f_Y(y)	& = 12y(1-y)^2 \qquad 0\leq y\leq 1 \\
\end{align*}
and
\begin{align*}
\expe(X) & = 12\int_0^1 x(1-x)^2\,dx = 2/5 \\
\expe(Y) & = 12\int_0^1 y(1-y)^2\,dy = 2/5 \\
\end{align*}

\it % << (c)
To compute $\expe(Y|X)$,
\begin{align*}
\expe(Y|X=x)
	& = \int_0^1 y\left(\frac{f_{X,Y}(x,y)}{f_X(x)}\right)\,dy \\
	& = \int_0^{1-x} y\left(\frac{24xy}{12x(1-x)^2}\right)\,dy \\
	& = \frac{24x}{12x(1-x)^2}\int_0^{1-x} y^2\,dy \\
	& =  \frac{24x}{12x(1-x)^2}\left[\frac{(1-x)^3}{3}\right] \\
	& = \frac{2}{3}(1-x)	
\end{align*}
so $\expe(Y|X) = 2(1-X)/3$.

\it % << (d)
\begin{align*}
\expe\big(\expe(Y|X)\big)
	& = \int_0^1 \expe(Y|X=x) f_X(x)\,dx \\
	& = \frac{2}{3}\int_0^1 (1-x) f_X(x)\,dx \\
	& = \frac{2}{3}\big(1-\expe(X)\big) = \frac{2}{3}\left(1-\frac{2}{5}\right) = \frac{2}{5} \\
	& = \expe(Y)
\end{align*}
\een
\mbox{}
\end{solution}
\end{example}

% exercises
\begin{exercise}
\begin{questions}
\question
Let $X$ and $Y$ be jointly continuous random variables having the following joint PDF,
\[
f_{X,Y}(x,y) = 
\begin{cases}
	\frac{21}{4}x^2y		& \quad x^2<y<1, \\
	0						& \quad\text{otherwise.}
\end{cases}
\]
\ben
\it Sketch the support of $f_{X,Y}$.
\it Find the marginal PDFs of $X$ and $Y$.
\it Find the mean and variance of $Y$.
\it Find the conditional PDF of $Y$ given $X=x$. 
\it Are $X$ and $Y$ independent? 
\it Find the conditional expectation of $Y$ given $X=x$. 
\it Find the conditional expectation of $Y$ given $X$. 
\it Verify that $\expe(Y)=\expe\big[\expe(Y|X)\big]$.
\een

\begin{answer}
\ben
\it % << (a)
The support of the joint PDF $f(x,y)$ is the set $\{(x,y): x^2 < y < 1\}$.  This is the region of the plane between the vertical lines $x=-1$ and $x=+1$, bounded above by the horizontal line $y=1$ and below by parabola $y=x^2$. In particular,
\bit
\it For fixed $x\in[-1,1]$, $f_{X,Y}(x,y)\neq 0$ only for $y\in[x^2,1]$.
\it For fixed $y\in[0,1]$, $f_{X,Y}(x,y)\neq 0$ only for $x\in[-\sqrt{y},+\sqrt{y}]$.
\eit

\it % << (b)
The marginal distributions are computed as follows:
\begin{align*}
f_X(x) 	
	& = \int_{-\infty}^{\infty} f(x,y)\,dy 
	= \int_{x^2}^1 \frac{21}{4}x^2y\,dy 
	= \frac{21}{4}x^2\left[\frac{y^2}{2}\right]_{x^2}^1 
	= \begin{cases} \frac{21}{8}x^2(1-x^4)	& -1<x<1, \\ 0 & \text{ otherwise.}\end{cases} \\ [2ex]
f_Y(y) 	
	& = \int_{-\infty}^{\infty} f(x,y)\,dx 
	= \int_{-\sqrt{y}}^{\sqrt{y}} \frac{21}{4}x^2y\,dx 
	= \frac{21}{4}y\left[\frac{x^3}{3}\right]_{-\sqrt{y}}^{\sqrt{y}}
	= \begin{cases} \frac{7}{2}y^{5/2} & 0< y< 1, \\ 0 & \text{ otherwise.}\end{cases} \\
\end{align*}

\it % << (c)
The expected value and variance of $Y$ are computed as follows:
\begin{align*}
\expe(Y)
	& = \int_{-\infty}^{\infty} y\,f_Y(y)\,dy 	= \int_0^1 y\left(\frac{7y^{5/2}}{2}\right)\,dy = \frac{7}{9}, \\
\expe(Y^2)
	& = \int_{-\infty}^{\infty} y^2\,f_Y(y)\,dy = \int_0^1 y^2\left(\frac{7y^{5/2}}{2}\right)\,dy = \frac{7}{11}, \\
\var(Y)
	& = \expe(Y^2) - \expe(Y)^2 = \frac{7}{11} - \frac{49}{81} = \frac{28}{891}.
\end{align*}

\it % << (d)
The conditional PDF of $Y$ given $X=x$ is
\[
f_{Y|X=x}(y) 	
	= \frac{f_{X,Y}(x,y)}{f_X(x)} 
	= \frac{(21/4)x^2y}{(21/8)x^2(1-x^4)} 	
	= \begin{cases} 
		\displaystyle\frac{2y}{1-x^4} 	& x^2\leq y\leq 1, \\
	 	0 								& \text{ otherwise}.
	 \end{cases}
\]	 

\it % << (e)
$X$ and $Y$ are clearly not independent, because the support of $f_{X,Y}$ is not a rectangular region, and moreover, the conditional PDF of $Y$ given $X=x$ depends on $x$.

\it % << (f)
The conditional expected value of $Y$ given that $X=x$ is
\begin{align*}
\expe(Y|X=x) 
	& = \int_{-\infty}^{\infty} y f_{Y|X=x}(y\,|\,x)\,dy \\
	& = \int_{x^2}^1 y \left(\frac{2y}{1-x^4}\right)\,dy 
	= \frac{2}{1-x^4}\left[\frac{y^3}{3}\right]_{x^2}^1
	= \frac{2(1-x^6)}{3(1-x^4)} 
\end{align*}	

\it % << (g)
The conditional expectation of $Y$ given $X$ is the random variable 
\[
\expe(Y|X)=\displaystyle\frac{2(1-X^6)}{3(1-X^4)}
\]

\it % << (h)
The expected value of $\expe(Y|X)$ is
\begin{align*}
\expe\big[\expe(Y|X)\big] 
	& = \int_{-\infty}^{\infty} \expe(Y|X=x)f_X(x)\,dx \\
	& = \frac{2}{3}\int_{-1}^{1} \left(\frac{1-x^6}{1-x^4}\right)\left(\frac{21}{8}x^2(1-x^4)\right)\,dx \\
	& = \frac{7}{4}\int_{-1}^{1} x^2(1-x^6)\,dx 
	= \frac{7}{9}.
\end{align*}	
Thus $\expe\big[\expe(Y|X)\big]=\expe(Y)$ as required.
\een
\end{answer}

% house for sale
\question
A man puts his house for sale, and decides to accept the first offer that exceeds the reserve price of $\pounds r$. Let $X_1,X_2,\ldots$ represent the sequence of offers received, and suppose that the $X_i$ are independent and identically distributed random variables, each having exponential distribution with rate parameter $\lambda$. 
\begin{parts}
\part
What is the expected number of offers received before the house is sold?% is $e^{\lambda r}$.
\begin{answer}
Let $N$ be the number of offers received before the house is sold. Then $\{N=k\}$ is the event that the first $k-1$ offers are at most $\pounds r$, each occurring independently with probability $F(r)$, and the $k$th offer exceeds $\pounds r$, which occurs with probability $1-F(r)$. Thus $N$ has \emph{geometric} distribution, with `probability of success' equal to $1-F(r)$ (where `success' corresponds to the sale of the house). Hence the expected number of offers received before the house is sold is
\[
\expe{E}(N) = \frac{1}{1-F(r)} = e^{\lambda r}.
\]
\end{answer}
\part 
What is the expected selling price of the house?
\begin{answer}
Let $F_S$ be the conditional CDF of $X_i$ given that $X_i>r$:
\[
F_S(x)  
	= \frac{\mathbb{P}(r < X_i \leq x)}{\mathbb{P}(X_i > r)}
	= \frac{F(x)-F(r)}{1-F(r)}
	= \begin{cases}
		1 - e^{-\lambda(x-r)} 	& x > r, \\
		0						& \text{otherwise.}
	\end{cases}
\]
A straightforward calculation yields $\mathbb{E}(X_i | X_i>r) = r + \displaystyle\frac{1}{\lambda}$.
\end{answer}
\end{parts}

\end{questions}
\end{exercise}
