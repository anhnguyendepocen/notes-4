% !TEX root = main.tex

%-------------------------------------------------
\section{Random points in the plane}\label{sec:joint}

Let $X$ and $Y$ be random variables on the same probability space.

% defn: joint distributions
\begin{definition}
\ben
\it The \emph{joint distribution} of $X$ and $Y$ is the function $\prob_{X,Y}$ defined on pairs of subsets of $\R$ by
\[
\prob_{X,Y}(C,D) = \prob(X\in C, Y\in D).
\]
\it The distributions $\prob_X(C)=\prob(X\in C)$ and $\prob_Y(D)=\prob(Y\in D)$ are called the \emph{marginal distributions} of $X$ and $Y$ respectively.
\een
\end{definition}

%The joint distribution of $X$ and $Y$ is completely determined by their joint CDF.
% defn: joint distributions
\begin{definition}
\ben
\it The \emph{joint CDF} of $X$ and $Y$ is the function $F_{X,Y}:\R^2\to[0,1]$ given by
\[
F_{X,Y}(x,y) = \prob(X\leq x,Y\leq y).
\]
\it The CDFs $F_X(x)=\prob(X\leq x)$ and $F_Y(y)=\prob(Y\leq y)$ are called the \emph{marginal CDFs} of $X$ and $Y$ respectively.
\een
\end{definition}

% defn: jointly discrete
\begin{definition}
\ben
\it $X$ and $Y$ are called \emph{jointly discrete} if the random vector $(X,Y)$ takes only countably many values in $\R^2$, in which case they are described by their \emph{joint PMF},
\[
\begin{array}{cccl}
f_{X,Y}:	& \R^2	& \to		& [0,1] \\
			& (x,y)	& \mapsto	& \prob(X=x,Y=y).
\end{array}
\]
\it The PMFs $f_X(x)=\prob(X=x)$ and $f_Y(y)=\prob(Y=y)$ are called the \emph{marginal PMFs} of $X$ and $Y$ respectively.
\een
\end{definition}

% defn: jointly continuous
\begin{definition}
\ben
\it $X$ and $Y$ are called \emph{jointly continuous} if their joint CDF can be written as
\[
F_{X,Y}(x,y) = \int_{-\infty}^x\int_{-\infty}^y f_{X,Y}(u,v)\,du\,dv \text{\qquad for all $x,y\in\R$,}
\]
for some integrable function $f_{X,Y}:\R^2\to[0,\infty)$ called the \emph{joint PDF} of $X$ and $Y$. 
\it The PDFs $f_X(x)=F'_X(x)$ and $f_Y(y)=F'_Y(y)$ are called the \emph{marginal PDFs} of $X$ and $Y$ respectively, where $F_X(x)$ and $F_Y(y)$ are their marginal CDFs. Note that
\[
f_X(x) = \int_{-\infty}^{\infty}f_{X,Y}(x,y)\,dy
\text{\qquad and\qquad}
f_Y(y) = \int_{-\infty}^{\infty}f_{X,Y}(x,y)\,dx.
\]
\een

\end{definition}

%-----------------------------
\subsection{Independence}

Recall that two events $A$ and $B$ are \emph{independent} if $\prob(A\cap B)=\prob(A)\prob(B)$.

% defn: independence
\begin{definition}
Two random variables $X$ and $Y$ are said to be \emph{independent} if the events $\{X\leq x\}$ and $\{Y\leq y\}$ are independent for all $x,y\in\R$.
\end{definition}

\begin{lemma}\label{lem:product_marginal_cdfs}
\ben
\it $X$ and $Y$ are independent if and only if $F_{X,Y}(x,y) = F_X(x)F_Y(y)$ for all $x,y\in\R$.
\it If $X$ and $Y$ are jointly discrete, they are independent if and only if $f_{X,Y}(x,y) =f_X(x) f_Y(y)$ for all $x,y\in\R$.
\it If $X$ and $Y$ are jointly continuous, they are independent if and only if $f_{X,Y}(x,y) = f_X(x)f_Y(y)$ for all $x,y\in\R$ and $\supp(f_{X,Y})$ is a rectangular region in $\R^2$.
\een
%\ben
%\it $X$ and $Y$ are independent if and only if 
%    \bit
%    \it $F_{X,Y}(x,y) = F_X(x)F_Y(y)$ for all $x,y\in\R$.
%    \eit
%\it If $X$ and $Y$ are jointly discrete, they are independent if and only if 
%    \bit
%    \it $f_{X,Y}(x,y) =f_X(x) f_Y(y)\quad\text{for all}\quad x,y\in\R$.
%    \eit
%\it If $X$ and $Y$ are jointly continuous, they are independent if and only if
%    \bit
%    \it $f_{X,Y}(x,y) = f_X(x)f_Y(y)$ for all $x,y\in\R$, and
%    \it the support of $f_{X,Y}$ is a rectangular region in $\R^2$.
%    \eit
%\een
\end{lemma}
\begin{proof}
The first two parts follow directly from the definitions. For the jointly continuous case,
\bit
\it because $X$ and $Y$ are independent, $F_{X,Y}(x,y)=F_X(x)F_Y(y)$, and differentiating both sides with respect to $x$ and $y$ we get $f_{X,Y}(x,y)=f_X(x)f_Y(y)$. 
\it If the value taken by $X$ affects the range of values taken by $Y$, then $Y$ clearly dependes on $X$. Thus for $X$ and $Y$ to be independent we need that
\[
\supp(f_{X,Y}) = \supp(f_X)\times \supp(f_Y),
\]
which is a rectangular region in $\R^2$.
\eit
\end{proof}

% example: discrete
\begin{exercise}\label{exc:joint_disc}
A fair die is rolled once. Let $\omega$ denote the outcome, and consider the random variables
\[
X(\omega) = \left\{\begin{array}{cl}
	1 & \text{ if $\omega$ is odd}, \\
	2 & \text{ if $\omega$ is even},
\end{array}\right. 
\text{\quad and\quad}
Y(\omega) = \left\{\begin{array}{cl}
	1 & \text{ if $\omega\leq 3$}, \\
	2 & \text{ if $\omega\geq 4$}.
\end{array}\right.
\]
Find the joint PMF of $X$ and $Y$. Are $X$ and $Y$ independent?
\begin{solution}
\[
\begin{array}{c|cc|c}
		& Y=1 	& Y=2	& f_X	\\ \hline
X=1		& 1/3	& 1/6	& 1/2	\\ 
X=2		& 1/6 	& 1/3	& 1/2 	\\ \hline	
f_Y		& 1/2	& 1/2	& 		\\ 
\end{array}
\]
$X$ and $Y$ are \emph{not} independent, because (for example) 
\[
\prob(X=1,Y=1)\neq\prob(X=1)\prob(Y=1).
\]
\end{solution}
\end{exercise}

% example: continuous
\begin{example}
Let $X$ and $Y$ be jointly continuous random variables with the following joint PDF:
\[
f_{X,Y}(x,y) = 
\left\{\begin{array}{ll}
	c(1-x)y	& \text{for } 0\leq y \leq x \leq 1, \\
	0		& \text{otherwise.}
\end{array}\right.
\]
\ben
\it Sketch the support of $f_{X,Y}$.
\it Are $X$ and $Y$ independent?
\it Find the marginal PDFs of $X$ and $Y$, and show that $c=24$.
%\it Show that $c=24$.
\een
\end{example}

\begin{solution}
\ben
\it % << (a)
$\supp(f_{X,Y})$ is the triangular region bounded by the lines $y=x$, $y=0$ and $x=1$.
\it % << (d)
The support of $f_{X,Y}$ is not a rectangular region in $\R^2$, so $X$ and $Y$ cannot be independent.
\it % << (c)
The marginal PDFs are:
\[\begin{array}{lll}
f_X(x) 	
	& = 24\int_0^x (1-x)y\,dy 
	= 24(1-x)\left[\frac{y^2}{2}\right]_0^x 
	& = \begin{cases}
		12x^2(1-x) & 0\leq x\leq 1, \\
		0			& \text{otherwise.}
		\end{cases} \\[3ex]
f_Y(y) 	
	& = 24\int_y^1 (1-x)y\,dx 
	= 24y\left[x - \frac{x^2}{2}\right]_y^1 
	& = \begin{cases}
		12y(1-y)^2 & 0\leq y\leq 1, \\
		0			& \text{otherwise.}
		\end{cases}
\end{array}\]
\it % << (b)
\bit
\it For fixed $x\in[0,1]$, we must have $0\leq y\leq x$.
\it For fixed $y\in[0,1]$, we must have $y\leq x\leq 1$.
\eit
\[
\int_{-\infty}^{\infty}\left(\int_{-\infty}^{\infty}f_{X,Y}(x,y)\,dy\right)\,dx
	= \int_0^1\int_0^x c(1-x)y\,dy\,dx 
	= \frac{c}{2}\int_0^1 x^2-x^3\,dx
%	= \frac{c}{2}\left(\frac{1}{3}-\frac{1}{4}\right)
	= \frac{c}{24}.
\]
By the law of total probability this must equal $1$, so $c=24$.
\een
\end{solution}

