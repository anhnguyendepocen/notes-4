% !TEX root = prime_factors.tex

\section{\en Lowest Common Multiple \cy Lluoswm Cyffredin Lleiaf }
\label{sec:lowest-common-multiple}

\en
The lowest/least common multiple (abbreviated to LCM) is the lowest number that is a multiple of two or more subject-numbers. 
\cy
Y lluosrif cyffredin lleiaf (wedi ei dalfyrru'n LlCLl) yw'r ��rhif lleiaf sy'n lluoswm o ddau neu fwy o rifau��.

\en
For example, the common multiples of 4 and 5 are 20, 40, 60, 80 and so on.
These are the numbers that are multiples of both 4 and 5.
The LCM is therefore 20, as this is the lowest of all the common multiples.
\cy
Er enghraifft, lluosrifau cyffredin 4 a 5 yw 20, 40, 60, 80...
Dyma'r rhifau sy'n lluosrifau o 4 a 5.
Y LlCLl felly yw 20, gan mai hwn yw'r lleiaf o'r holl luosrifau cyffredin.

\begin{example}
\en Find the LCM of 6 and 10.
\cy Canfydda luosrif cyffredin lleiaf 6 a 10.
\bi
\begin{solution}

\en If we write out the 6 times table, we get:
\cy Os ysgrifennwn ni dabl lluosi rhif 6, cawn: 

6, 12, 18, 24, 30, 36, 42, 48, 54, 60, 66, 72.

\en If we write out the 10 times table, we get: 
\cy Os ysgrifennwn ni dabl lluosi rhif 10, cawn:
 
10, 20, 30, 40, 50, 60, 70, 80, 90, 100, 110, 120.

\en So the LCM of 6 and 10 is 30.
\cy Felly LlCLl 6 a 10 yw 30.
\bi
\end{solution}
\end{example}

\begin{example}
\en Find the LCM of 12 and 36.
\cy Canfydda LlCLl 12 a 36.
\bi
\begin{solution}

\en If we write out the 12 times table, we get: 
\cy Os ysgrifennwn ni dabl lluosi rhif 12, cawn: 

\bi 12, 24, 36, 48, 60, 72, 84, 96, 108, 120, 132, 144 ...

\en If we write out the 36 times table, we get:
\cy Os ysgrifennwn ni dabl lluosi rhif 36, cawn: 

36, 72, 108, 144, 180, 216 ...

\en From the times tables above, we can see that all the numbers in the 36 times table appear in the 12 times table.
\cy O'r tablau lluosi uchod, gwelwn fod yr holl rifau sydd yn nhablau lluosi 36 yn ymddangos yn nhablau lluosi 12 hefyd.


\en The LCM of 12 and 36 is 36.
\cy LlCLl 12 a 36 yw 36.

\en Sometimes we may be asked to find the LCM of more than two numbers. The process is exactly the same, although this does increase the difficulty slightly.
LlCLl 12 a 36 yw 36.
\cy Weithiau, efallai y bydd gofyn i ni ganfod LlCLl mwy na dau rif. Mae'r broses yn union yr un fath, er bod hyn yn ei gwneud fymryn yn anoddach.
\bi
\end{solution}
\end{example}

\begin{example}
\en Find the LCM of 3, 4 and 5.
\cy Canfydda LlCLl 3, 4 a 5.

\begin{solution}

\en If we write out the 3 times table, we get: 
\cy Os ysgrifennwn ni dabl lluosi rhif tri, cawn: 

3, 6, 9, 12, 15, 18, 21, 24, 27, 30, 33, 36, 39, 42, 45, 48, 51, 54, 57, 60.

\en If we write out the 4 times table, we get: 
\cy Os ysgrifennwn ni dabl lluosi rhif pedwar, cawn: 

4, 8, 12, 16, 20, 24, 28, 32, 36, 40, 44, 48, 52, 56, 60, 64, 68, 72, 76, 80.

\en If we write out the 5 times table, we get: 
\cy Os ysgrifennwn ni dabl lluosi rhif 5, cawn: 

5, 10, 15, 20, 25, 30, 35, 40, 45, 50, 55, 60, 65, 70, 75, 80, 85, 90, 95, 100.

\en Note that in order to find a common multiple, we had to write out far more than 12 numbers in each times table.
\cy Mae'n werth nodi, er mwyn canfod lluosrif cyffredin, roedd yn rhaid i ni ysgrifennu llawer mwy na 12 rhif ym mhob tabl lluosi.

\en We can see that the LCM of 3, 4, and 5 is 60.
\cy Gallwn weld mai LlCLl 3, 4, a 5 yw 60.

\end{solution}
\end{example}

