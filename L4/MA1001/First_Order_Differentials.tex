% !TEX root = main.tex

%------------------------------------------------
\chapter{First Order Differentials}

A first order (ordinary differential equations) is an equation of the form : $$\frac{dy}{dt} = f(t,y).$$

Here any solution $y$ is a real or complex valued function of a single real variable and $f$ is a function of 2 variables.

In Newtonian notation the same equation would be written as : $$y'(t) = f(t,y(t)).$$
\section{Separable Equations}

A separable equation is an equation of the form : $$\frac{dy}{dt} = g(t)h(y).$$
Suppose that $h$ never takes the value 0. Then we can write the equation as : $$\frac{1}{h(y)}\frac{dy}{dt}  = g(t).$$
We can also (at least on principle) find a function $F$ whose derivative is $\frac{1}{h}$ : $$F' = \frac{1}{h},$$
hence : $F'(y(t))\frac{dy}{dt} = g(t) \implies \frac{d}{dt}[F(y(t))] = g(t).$

\bigskip

\begin{example}  Solve $$\frac{dy}{dt} = (1+y^2)(1+\alpha{y})$$ where $\alpha \geq 0$ is constant. \newline
\begin{solution} Write the equation as $$\frac{1}{1+y^2}\frac{dy}{dt} = 1 + \alpha{t}.$$
Here we need $F'(y) = \frac{1}{1+y^2}$ so we choose : $$F(y) = tan^{-1}(y)$$ and get : $$\frac{d}{dt}(tan^{-1}(y(t))) = 1 + \alpha{t},$$
and so by integration $$tan^{-1}(y(t)) = t +\alpha{\frac{t^2}{2}} + \beta$$ where $\beta$ is constant.
Hence : $y(t) = tan( t +\alpha\frac{t^2}{2} + \beta).$
Remark : The solution of this equation always blows up in finite time even when $\alpha = \beta = 0 $. In this case the solution blows up at $t = \frac{\pi}{2}$.
\end{solution}
\end{example}
\bigskip

\begin{example}  Find the solution of $$\frac{dy}{dt} = 2\sqrt[]{y}.$$
\begin{solution} Divide by $2\sqrt[]{y}$ to get $$ \frac{1}{2\sqrt[]{y}}\frac{dy}{dt} = 1.$$
For this case : $$F'(y) = \frac{1}{2\sqrt[]{y}}$$ and so we choose $F(y) + \sqrt[]{y}$ and get $$\frac{d}{dt}(\sqrt[]{y(t)} = 1.$$
So, on integration $$\sqrt[]{y(t)} = t + c $$ where $c$ is a constant.
This gives $y(t) = (t + c)^2.$
Suppose we want $y(0) = 0$ we get : $$0 = y(0) = (0 + c)^2 = c^2.$$
Giving $c = 0$. Thus $y(t) = t^2.$

Warning : This is not the only solution of $\frac{dy}{dt} = 2\sqrt[]{y}$ which satisfies $y(0) = 0$, as we said previously. We could also have $y(t) = 0$ for all $t$.
\end{solution}
\end{example}

\bigskip 

\begin{example}  Solve the equation : $$\frac{dy}{dt} = m(1 - m).$$
\begin{solution} Suppose $ m \neq 0$, $ m \neq 1$ Then $$\frac{1}{m(1-m)}\frac{dm}{dt} = 1.$$
We need to find a function $F$ such that :
$$F'(m) = \frac{1}{m(1-m)} = \frac{1}{m} + \frac{1}{1-m}.$$
A suitable $F$ is: $$F(m) = \log(m) - \log(1-m) = \log(\frac{m}{1-m})$$ and so our equation is : $$\frac{d}{dt}[\log(\frac{m}{1-m})] = 1 \implies \log(\frac{m}{1-m}) = t + c$$ where $c$ is a constant, thus $$\frac{m(t)}{1-m(t)} = \exp(t + c) = \alpha \exp(t)$$ where $\alpha = \exp(c)$ is also a constant. Suppose that $m(0) = \mu$ then : $$\frac{\mu}{1-\mu} = \alpha.$$
The equation for $m(t)$ can be rearranged as $$\mu(t) = \frac{\alpha \exp(t)}{1 + \alpha \exp(t)} = \frac{1}{\frac{1}{\alpha} \exp(-t) + 1} = \frac{1}{(\frac{1 - \mu}{\mu}) \exp(-t) + 1} = \frac{\mu}{(1-\mu) \exp(-t) = \mu}.$$
\end{solution}
\end{example}

\bigskip 

\begin{remark}
\begin{itemize}
\item If we take $m(0) = \mu = 0$ we get $m(t) = 0$ for all $t$ then : \newline
If $0 < \mu , 1$ then $(1 - \mu)\exp(-t) > 0$ and hence $$ 0 , m(t) < \frac{\mu}{0 + \mu} = 1$$ Thus $$\frac{dm}{dt} = m(1-m) > 0 \mbox{ also } \lim m(t) = 1.$$

\item If we take $m(0) = \mu = 1$ we get $m(t) = 1$ for all $t$ then :
\newline
If $\mu > 1$ then the expression $$m(t) = \frac{\mu}{\mu(1-\exp(-t))+ \exp(-t)}$$ tells us that $m(t) > 0$ whilst the expression $$m(t) = \frac{\mu}{(1-m)\exp(-t) + \mu}$$ tells us that $m(t) > 1$ for all $t \geq 0$. Also $\frac{dm}{dt} = m(1-m) < 0$, so $m$ is strictly decreasing. Also $\lim m(t) = 1$.

\end{itemize}
\end{remark}
\section{Homogeneous Equations }

A homogeneous equation is an equation of the form : $$\frac{dy}{dt} = f(\frac{y}{t}).$$
We introduce a new function $z = \frac{y}{t}$; more precisely $z(t) = \frac{y(t)}{t}$. Hence $y(t) = tz(t)$ which gives $$f(z) = \frac{dy}{dt} = z(t) + t\frac{dz}{dt}.$$
Rearranging yields: $$\frac{dz}{dt} = \frac{f(z) - z}{t} = \frac{1}{z}(f(z) - z)$$ which is now separable.

\bigskip

\begin{example} Solve the equation : $$t^2\frac{dy}{dt} = ty + t^2 + y^2$$ and examine the behavior of the solution as $t$ tends to 0. 

\begin{solution} Divide by $t^2$ to get $$\frac{dy}{dt} = \frac{y}{t} + 1 +(\frac{y}{t})^2 = f(\frac{y}{t})$$ where $f(z) = z + 1 + z^2$. We let $y = tz$ and get $$\frac{dz}{dt} = \frac{1}{t}(f(z)-z) = \frac{1}{t}(1+z^2)\implies \frac{1}{1+z^2}\frac{dz}{dt} = \frac{1}{z},$$
so $$\frac{1}{1 +z^2}dz = \frac{1}{t}dt.$$
Integrate both sides to obtain $tan^{-1} = \log(t) + c$ where $c$ is a constant of integration. This yields $$z = tan(c + \log(t)) \implies y - t*tan(c+\log(t)).$$
Put $c = \log\alpha$ where $\alpha = \exp(c)$ so that $$y = t*tan(\log\alpha + \log t) = t*tan(\log(\alpha t )).$$ This blows up whenever $\log(\alpha t) = (2k + 1)\frac{\pi}{2}$, $k\in \mathbb{Z}$ i.e $$\alpha t = \exp((2k + 1)\frac{\pi}{2}) \mbox{,} \mbox{ or } t = \frac{1}{\alpha}\exp((2k + 1)\frac{\pi}{2}).$$
Letting $k$ decrease to $-\infty$ through the negative integers we see that these are infinitely many points at which $y$ blows up, in any neighborhood of zero. 
\end{solution}
\end{example}

\bigskip

\begin{example}  Solve the homogeneous equation : $$\frac{dy}{dt} = \frac{y}{t}^{\frac{1}{3}}.$$
\begin{solution} Put $y = tz, f(z) = z^{\frac{1}{3}}$, and obtain $$\frac{dz}{dt} = \frac{1}{t}(f(z) - z) = \frac{1}{t}(z^{\frac{1}{3}} - z).$$ This yields $$\frac{1}{z^\frac{1}{3} - z}dz = \frac{1}{t}dt.$$
Alternatively, the original equation is already separable, as $y^{\frac{-1}{3}}dy = t^\frac{-1}{3}dt$. By integration : $$y^\frac{2}{3} = t^\frac{2}{3} + c$$ where $c$ is a constant of integration, thus $$ y = (c + t^\frac{2}{3})^\frac{3}{2}.$$
\end{solution}
\end{example}

\section{Linear Equations}
A first-order differential linear equation is an equation of the form : $$\frac{dy}{dt} = a(t)y+b(t)$$ where $a(t)$ and $b(t)$ are given functions of the dependant variable $t$. For linear equations we can always write the solution $y$ explicitly as a function.

\subsection*{Solution Procedure}
\begin{enumerate}
\item Choose a function $p(t)$ such that $$\frac{dp}{dt} = a(t)$$
\item Observe that $$\frac{d}{dt}(ye^{-p(t)}) = \frac{dy}{dt}e^{-p(t)} + ye^{-p(t)} = (\frac{dy}{dt} - a(t)y)e^{-p(t)} = b(t)e{-p(t)}$$
\item Integrate the equation $$\frac{d}{dt}(ye^{-p(t)}) = b(t)e^{-p(t)}$$ This integrates to $$ye^{-p(t)} = \int b(t)e{-p(s)}ds + c$$ where $c$ is a constant
By doing this we get : $$y = e^p(t)(c+ \int b(s)e^{-p(s)}ds$$
\end{enumerate}
\begin{example} Solve the differential equation $$\frac{dy}{dt} = ky +\sin(t)$$ subject to the condition $y(0)=0$.
\begin{solution} Choose $p = \int kdt = kt$ where $a(t) = k$ and $b(t) = \sin(t)$.
Therefore $$\frac{d}{dt}(e^{-kt}y) = e^{-kt}\sin(t).$$ There are at least two ways of integrating the right hand side of the equation. One is to integrate by parts and the other is to use complex exponentials, therefore : $$\sin(t) = Im(e^{it}).$$ Hence : $$\int e^{-kt}\sin(t) dt \implies \int e^{-kt} Im(e^{-it}dt = Im(\int e^{-kt + it}dt) \implies Im(\frac{e^{(i-k)t}}{(i-k)}$$ $$= Im(\frac{(-i-k)(\cos(t) + \sin(t))*e^{-kt}}{1+k^2} = Im(\frac{(-k\cos(t) +\sin(t) +i(-\cos(t) - k\sin(t)))e^{-kt}}{1+k^2})$$ $$= \frac{-(\cos(t) + k\sin(t))e^{-kt}}{1+k^2}.$$
Therefore :
$$e^{-kt}y =   -(\frac{-(\cos(t) + k\sin(t))e^{-kt}}{1+k^2})$$ where $c$ is a constant and when $t=0$ and $y=0$ so : $$c - \frac{1}{1+k^2} = 0 \implies c = \frac{1}{1+k^2}$$
Thus the solution is $$y = \frac{e^{kt} - (\cos(t) + k\sin(t))}{1+k^2}.$$
\end{solution}
\end{example}
\bigskip

\begin{example} Solve the linear differential equation $$t^2\frac{dy}{dt} + 2ty = e^t$$ subject to $y(1) = 1.$ Is there a solution subject to the condition $y(0)=1$?
\begin{solution}
We can write the equation in the form $$\frac{dy}{dt}= \frac{-2}{t}y +\frac{e^t}{t^2}.$$ Then we find the integrating factors as before however in this case we can simplify the equation by inspection : $$t^2\frac{dy}{dt} + 2ty = \frac{d}{dt}(t^2y) = e^t.$$
Then by integration : $$t^2y = e^t + c \implies y = \frac{e^t + c}{t^2}$$ where $c$ is a constant.
The initial condition $y(1) = 1$ gives you $$ 1 = \frac{e + c}{1} \implies c = 1-e.$$ Thus $$y = \frac{e^t + 1 - e}{t^2}.$$ There is no solution which takes the value 1 when $t=0$! This is because when we write the equation as $\frac{dy}{dt}= f(t,y)$ we have $$\frac{dy}{dt} = \frac{-2y}{t} + \frac{e^t}{t^2}$$ and this function is badly behaved as $t \rightarrow 0$. The initial conditions can not be imposed at a point where the right hand side of the differential equation blows up.
\end{solution}
\end{example}

\bigskip

\begin{example}  Solve the equation $$(t\log(t))\frac{dy}{dt} + y = 3t^3.$$
\begin{solution}
First we must find an integrating factor which is $$\exp( \int \frac{1}{t\log(t)} dt).$$ Then by integration by substitution where $u = \log(t)$ and $\frac{du}{dt} = \frac{1}{t}$ we have that $$\int \frac{1}{t\log(t)} dt \implies \int \frac{1}{u}du = \log(u) \implies \log(\log(t)).$$
Now the integrating factor is $\exp(\log(\log(t))$ which is $\log(t)$ and therefore : $$\log(t)\frac{dy}{dt} + \frac{y}{t} = 3t^2 \implies \frac{d}{dt}[\log(t)y] = 3t^2.$$ Then by integration you get $$\log(t)y = \int3t^2 dt = t^3 + c$$ where $c$ is a constant. Thus by rearranging $$y = \frac{t^3 + c}{\log(t)}.$$
\end{solution}
\end{example}

\bigskip

\section{Bernoulli Equations}
A Bernoulli equation is an equation in the form $$\frac{dy}{dt} + a(t)y = b(t)y^n.$$
Where when $n = 0$ we have a linear equation and when $n = 1$ we have  linear and separable equation. We choose an integration factor $p(t)$ such that $$\frac{p'(t)}{p(t)} = a(t) \mbox{ so that } \frac{dy}{dt} + \frac{p'(t)}{p(t)}y = b(t)y^n.$$
Now we multiply both sides by $p(t)$ to get $$p(t)\frac{dy}{dt} + \frac{dp}{dt}y = p(t)b(t)y^n.$$ Then by differentiation $$\frac{d}{dt}(p(t)y(t0 = \frac{b(t)}{p(t)^{n-1}}(p(t)y(t))^n.$$ We introduce a new function $z$ by the formula $z(t) = p(t)y(t)$ which satisfies $$\frac{dz}{dt} = q(t)z^n \mbox{ where } q(t) = \frac{b(t)}{p(t)^{n-1}}.$$

\begin{example}
Solve the equation $$\frac{dy}{dt} + \frac{2}{t}y = \exp(t)y^2.$$
\begin{solution}
Multiply both sides by $t^2$ to obtain $$t^2\frac{dy}{dt} + 2ty = t^2\exp(t)y^2 \implies \frac{d}{dt}(t^2y) = \frac{1}{t^2}\exp(t)(t^2y)^2.$$ Let $z = t^2y$ such that $$\frac{dz}{dt}=\frac{1}{t^2}\exp(t)z^2$$ which is now a separable equation. This separates as $$\frac{1}{z^2}\frac{dz}{dt} = \frac{1}{t^2}\exp(t) \mbox{ or } \frac{d}{dt}(\frac{-1}{z}) = \frac{1}{t^2}\exp(t)$$ Any further progress depends on integrating the right hand side, probably with an incomplete gamma-function.
\end{solution}
\end{example}
\bigskip

\begin{example}
Solve the equation $$\frac{dy}{dt} = t\exp(t^2-y).$$
\begin{solution}
Thanks to the property $\exp(t^2-y) = \frac{\exp(t^2)}{\exp(y)}$ this is a separable equation. We rearrange it as $$\exp(y)\frac{dy}{dt} = t\exp(t^2) \implies \frac{d}{dt}\exp(y)) = \frac{d}{dt}(\frac{1}{2} \exp(t^2.)$$
Thus $$\exp(y) = \frac{1}{2} \exp(t^2) + c$$ where $c$ is a constant. Finally $$y = \log(c + \frac{1}{2} \exp(t^2)).$$ For the special case when $c = 0$ we get $$y = \log(\frac{1}{2} \exp(t^2)) \implies \log(\frac{1}{2}+t^2 = t^2 - \log(2).$$
\end{solution}
\end{example}


%------------------------------------------------
\endinput
