% !TEX root = main.tex

%------------------------------------------------
 \chapter{Second Order Equations}
A second order linear differential equation is an equation of the following form : $$\frac{d^2y}{dt^2} +a_1(t)\frac{dy}{dt} + a_0(t)y = f(t).$$ The functins $a_1(*)$, $a_0(*)$ and $f(*)$ are supposed to be known and we want to find all of the solutions to $y$.
\begin{example}
The Legendre equation is $$\frac{d^2y}{dt^2} - \frac{2t}{1-t^2}\frac{dy}{dt} + \frac{n(n+1)}{1-t^2}Y = 0.$$
The equation has a solution $P_n(t)$, the Legendre polynomial of degree $n$.
\begin{itemize}
\item For $n = 0$ we can take $P_0(t)$ for all $t$
\item For $n = 1$ we let $P_1(t) = at+b$. We get $$\frac{-2t}{1 - t^2}a + \frac{2}{1-t^2}(at+b) = 0$$ for all $t\in\mathbb{R}\backslash \{\pm 1\}$ This gives $b=0$ and we can take $a = 1$ to get $P_1(t) = t$
\end{itemize}
\end{example}
\begin{example}
Bessel's equation of order $n$ is : $$\frac{d^2y}{dt^2} + \frac{1}{t}\frac{dy}{dt}+(1 - \frac{n^2}{t^2})y = 0.$$
The solution are Bessel functions, $J_n(t)$ and $Y_n(t).$
\end{example}
\section{Equations with Constant Coefficients}
$$\frac{d^2y}{dt^2}+a_1\frac{dy}{dt}a_0y = f(t)$$
In order to solve the above equation we adopt a two step solution strategy: \begin{enumerate}
\item Find all the solution of the equation with $f = 0$. The equation when $f = 0$ is called the homogeneous equation
\item Develop a formula called the variation of parameters formula for the case $f \not = 0$, using the solutions of the homogeneous equation
\end{enumerate}
\subsubsection*{Solution Method for $\mathbf{f = 0}$}
We look for a solution of the form $y(t) = A\exp(\mu t)$, where $A$ and $\mu$ are constants. We have: $$\frac{dy}{dt} = \mu A\exp(\mu t) \quad \mbox{      and      } \quad \frac{d^2y}{dt^2} = \mu^2 A \exp(\mu t).$$
Substituting into the differential equation gives: $$(\mu^2 + a_1\mu + a_0) * A\exp(\mu t) = 0.$$
Since $\exp(\mu t ) \not = 0$ and we do not want the trivial solution which comes from choosing $ A = 0$, we have $\mu^2 + a_1\mu +a_0 = 0$ this is called the auxiliary quadratic. The possibilities are:
\begin{itemize}
\item Two distinct roots
\item One double root
\end{itemize}
If $a_1$ and $a_0$ are real there is an alternative dichotomy: 
\begin{itemize}
\item Two real roots (which may or may not coincide)
\item A complex conjugate pair of distinct roots
\end{itemize}
\subsubsection*{Case 1 : Distinct Roots $\mathbf{\mu_1 \not = \mu_2}$}
$$\mu^2 +a_1\mu +a_0 = (\mu - \mu_1)(\mu-\mu_2).$$ We have the solutions $$y_j(t) = A_j \exp(\mu_j t) \quad j = 1, 2.$$ For completely arbitary constants $A_1$, $A_2$. Suppose $y(t) = y_1(t) + y_2(t)$ then $$\frac{dy}{dt} = \frac{dy_1}{dt} + \frac{dy_2}{dt} \quad \mbox{   and   } \quad  \frac{d^2y}{dt^2} = \frac{d^2y_1}{dt^2} + \frac{d^2y_2}{dt^2}.$$
So, $$\frac{d^2y}{dt^2} + a_1\frac{dy}{dt} + a_0y = [\frac{d^2y_1}{dt^2} + a_1\frac{dy_1}{dt} +a_0y_1] +  [\frac{d^2y_2}{dt^2} + a_1\frac{dy_1}{dt} + a_0y_2].$$
 \begin{theorem}

Suppose that the auxiliary quadratic has a distinct root $\mu_1 \not = \mu_2$. Then every solution of the homogeneous equation has the form $$y(t) = A_1 \exp(\mu_1t) + A_2 \exp(\mu_2t)$$ for some $A_1$, $A_2 \in \mathbb{C}.$
\end{theorem}
\begin{example}
$$\frac{d^2y}{dt^2} - (\mu_1+\mu_2)\frac{dy}{dt} + \mu_1\mu_2y = 0$$
such that $y(0) = y(0) \mbox{ and } y'(0) = v_0.$
\begin{solution}

We know $y(t) = A_1 \exp(\mu_1t) + A_2 \exp(\mu_2t)$ and we choose $A_1$, $A_2$ so that:
$$
\begin{cases}
y_0 = y(0) = A_1 + A_2 \\
v_0 = y'(0) = A_1\mu_1 +A_2\mu_2
\end{cases} $$
Hence $$\mu_2y_0 - v_0 = A_1(\mu_2 - \mu_1) \implies$$
$$
\begin{cases}
A_1 = \frac{\mu_2y_0 - v_0}{\mu_2 - \mu_1} \\
A_2 = y_0 - A_1 = \frac{v_0 - \mu_1y_0}{\mu_2 - \mu_1}
\end{cases} $$
Thus $$y_1(t) = \frac{\mu_2y_0 - v_0}{\mu_2 - \mu_1} \exp(\mu_1t) + \frac{v_0 - \mu_1y_0}{\mu_2 - \mu_1} \exp(\mu_2t) = y_0\bigg(\frac{\mu_2\exp(\mu_1t) - \mu_1\exp(\mu_2t)}{\mu_2-\mu1}\bigg) + v_0\bigg(\frac{\exp(\mu_2t) - \exp(\mu_1t)}{\mu_2-\mu_1}\bigg).$$
\end{solution}
\end{example}
\begin{example}
In the previous example let $\mu_1$ be fixed and find $\lim_{\mu_2 \to \mu_1} y(t)$.
\begin{solution}
First we compute $$\lim_{\mu_2 \to \mu_1}\bigg[\frac{\exp(\mu_2t) - \exp(\mu_1t)}{\mu_2 - \mu_1}\bigg] = t\exp(\mu t)\lim_{\mu_2 \to \mu_1}\bigg[\frac{\exp((\mu_2 - \mu_1)t - 1}{(\mu_2 - \mu_1)t}\bigg].$$
Now set $ h = \mu_2 - \mu_1$. Thus : $$t\exp(\mu_1t)\lim_{h \to 0}\bigg[\frac{\exp(h) - 1}{h}\bigg]$$ $$= t\exp(\mu_1)\lim_{h \to 0}\bigg[\frac{\exp(h) - \exp(0)}{h}\bigg] = t\exp(\mu_1t)\exp'(0).$$ 
Where $\exp'(0) = 0$. Hence $$\lim_{\mu_2 \to \mu_1}\bigg(\frac{\exp(\mu_2t) - \exp(\mu-1t)}{\mu_2 - \mu_1}\bigg) = t\exp(\mu_1t).$$
Next we compute:
$$\lim_{\mu_2 \to \mu_1}\bigg(\frac{\mu_2\exp(\mu_1t) - \mu_1\exp(\mu-2t)}{\mu_2 - \mu_1}\bigg).$$
We observe that:
$$\frac{\mu_2\exp(\mu_1t) - \mu_1\exp(\mu-2t)}{\mu_2 - \mu_1} = \bigg[\frac{\mu_1(\exp(\mu_2t) - \exp(\mu_1t)) + (\mu_2 - \mu_1) \exp(\mu_1t)}{\mu_2 - \mu_1}\bigg]$$ $$= \exp(\mu_1t) - \mu_1 \bigg[\frac{\exp(\mu_2t -\exp(\mu_1t)}{\mu_2 - \mu_1}\bigg].$$
Thus $$\lim_{\mu_2 \to \mu_1}\bigg(\frac{\mu_2\exp(\mu_1t) - \mu_1\exp(\mu-2t)}{\mu_2 - \mu_1}\bigg) = \exp(\mu_1t) - \mu_1t\exp(\mu_1t).$$
Hence $$\lim_{\mu_2 \to \mu_1} y(t) = y_0(1 - \mu_1 )\exp(\mu_1t) + v_ t \exp(\mu_1t).$$
Let $$z(t) = y_0(1 - \mu_1t )\exp(\mu_1t) + v_0 t \exp(\mu_1t).$$ Then
$$
\begin{cases}
z(0) = y_0 \\
z'(t) = y_0(-\mu_1 + \mu_1(1 - \mu_1t)) \exp(\mu_1t) + v_0(1+\mu_1t)\exp(\mu_1t)
\end{cases}
$$
giving $z'(0) = v_0.$
Let $z(t) = (At+B)\exp(\mu_1t)$, where $A$ and $B$ are constants. We want to check that $$\frac{d^2z}{dt^2}-2\mu_1\frac{dz}{dt}+{\mu^2}_1z = 0.$$
We have $$z(t) = (At+B)\exp(\mu_1t) \implies \frac{dz}{dt} = (A +\mu_1(At+B))\exp(\mu_1t) = (\mu_1At + (A + \mu_1 B))\exp(\mu_1t)$$ $$\implies \frac{d^2z}{dt2} = (\mu_1At + (\mu_1A +\mu_1(A+\mu_1B))\exp(\mu_1).$$
Hence
$$\frac{d^2z}{dt^2}-2\mu_1\frac{dz}{dt}+{\mu^2}_1z = (({\mu^2}_1A - 2\mu_1*\mu_1A +{\mu^2}_1A)t + 2\mu_1A +{\mu^2}_1B - 2\mu_1(A+\mu_1B) +{\mu^2}-1B)\exp(\mu_1t) = 0 \mbox{ as required.}$$
\end{solution}
\end{example}
\begin{theorem}
Every solution of the equation $$\frac{d^2y}{dt^2}-2y\frac{dy}{dt} +\mu^2y = 0$$ has the form $$y(t) = (At+B)\exp(\mu t)$$ for some appropriate constants $A$, $B$.
\end{theorem}
\begin{example}
Find the general solution of $$\frac{d^2y}{dt^2} + \omega^2y = 0$$ where $\omega>0$ is constant.
\begin{solution}
We look for a solution of the form $$y(t) = A\exp(\mu t)$$ and obtain the auxiliary quadratic $\mu^2 + \omega^2 = 0$, which has roots $\mu_1 = i\omega$, $\mu_2 = -i\omega$. Thus the general solution has the form $$y(t) = A\exp(i\omega t) + B\exp(-i\omega t)$$
$$ = A(\cos(\omega t ) +i\sin(\omega t)) + B(\cos(\omega t) - i\sin(\omega t))$$ $$= (A+B)\cos(\omega t) +i(A-B)\sin(\omega t).$$
Suppose $y(0) = y_0 \in \mathbb{R}$ and $y'(0) = v_0 \in \mathbb{R}.$ Then 
$$ 
\begin{cases}
y(0) = A+B = y_0 \\
y'(0) = i\omega(A-B) = V_0
\end{cases} $$
Thus $y(t) = y_0\cos(\omega t) + v_)\frac{\sin(\omega t)}{\omega}.$
\end{solution}
\end{example}
\begin{remark}
Observe that $\lim_{\omega \to 0}\cos(\omega t) = \cos(0) = 1 \mbox{ and } \lim_{\omega \to 0}\frac{\sin(\omega t)}{\omega} = t$. When $\omega = 0$, we get $$y(t) = y_0 + v_0t$$ which does indeed solve $\frac{d^2y}{dt^2} = 0$.
\end{remark}
\begin{definition}
Let $y_1$, $y_2$ be solutions of $$\frac{d^2y}{dt^2} + a_1(t)\frac{dy}{dt} +a_0(t)y=0.$$ We say that $y_1$ and $y_2$ are linearly independent if and only if neither solution can be written as a constant multiple of the other.
\end{definition}
\begin{example}
In the previous example, $$y_1(t) = \cos(\omega t) \mbox{ and }  y_2 = \frac{\sin(\omega t)}{\omega}$$ are linearly independent: so are $$y_1(t) = \exp(i \omega t) \mbox{ and } y_2(t) = \exp(-i \omega t)$$  $$y_1(t) = \cos(\omega t) \mbox{ and } y_2(t) = \exp(i\omega t).$$ However if $y_1t = \cos(\omega t)$ and $y_2(t) = 17 \cos(\omega t)$, are not linearly independent.
\end{example}
\begin{definition}
Let $y_1$ and $y_2$ be two differentiable functions. The Wronskian (determinant) of $y_1$ and $y_2$ is the function $$W = \det \begin{bmatrix}
y_1 \hspace{4pt} y_2 \\
y'_1 \hspace{4pt} y'_2 \\
\end{bmatrix}
= y_1y'_2 - y'_1y_2$$
\end{definition}
\subsubsection*{Objective}
We want to find a formula for the general solution of $$\frac{d^2y}{dt^2} + a_1\frac{dy}{dt} + a_0y = f$$ where the function $f$ on the right hand side is non-trivial. We assume that we know two lineralry independent solutions $y_1$ and $y_2$ of the homogeneous equation $$\frac{d^2y}{dt^2}j + a_1\frac{dy}{dt}j + a_0yj = 0 \mbox{ for } j = 1,2.$$ We can look for a solution in the form $$y(t) = A_1(t) +A_2(t)y_2(t).$$ Differentiating, $$y'(t) = A_1(t)y'_1(x) + A_2(t)y'_2(t) + A'_1(t)y_1(t) + A'_2(t)y_2(t). $$
We impose the condition $A'_1y_1 +A'_2y_2 = 0$. Then we obtain by the second derivative $$y''(t) = A_1(t)y''_1(t) +A_2y''_2(t) + A'_1(t)y'_1(t) +A'_2(t)y'_2(t).$$ Combining the expressions for $y$, $y'$ and $y''$, we obtain $$y'' + a_1y +a_0y = A_1(y''_1 + a_1y'_1 + a_0y1) + A_2(y''_2 + a_1y'_2 + a_0y2) +A'_1y'_1+A'_2y'_2 = A'_1y'_1+A'_2y'_2 $$ because $y_1$ and $y_2$ solve the homogeneous equation. Using the original formula we see that our second equation for $A_1$ and $A_2$ is $$A'_1y'_1+A'_2y'_2 = f.$$ We write the equation for $A_1$ and $A_2$ as $$\begin{bmatrix}
y_1 \hspace{10pt} y_2 \\
y'_1 \hspace{10pt} y'_2 \\
\end{bmatrix}
\begin{bmatrix}
A'_1 \\
A'_2 \\
\end{bmatrix}
= \begin{bmatrix}
0 \\
f \\
\end{bmatrix}
$$
which yields
$$ \begin{bmatrix}
A'_1 \\
A'_2 \\
\end{bmatrix} = \frac{1}{W}
\begin{bmatrix}
y'_2 \hspace{10pt} -y_2 \\
-y'_1 \hspace{10pt} y_1 \\
\end{bmatrix}
\begin{bmatrix}
0 \\
f \\
\end{bmatrix}
= \frac{1}{W}
\begin{bmatrix}
-y_2f \\
y_1f \\
\end{bmatrix}
$$ This means $$ \begin{cases}
A'_1 = \frac{-y_2 f}{W}\\
A'_2 = \frac{y_1 f}{W} \end{cases}$$ giving
$$
\begin{cases}
A_1(t) = \int_{t_0}^{t} \frac{-y_2(x) f(x)}{W(x)} dx + \alpha_1 \\

\bigskip

A_2(t) = \int_{t_0}^{t} \frac{y-1(x) f(x)}{W(x)} dx +\alpha_2\\
\end{cases}$$
Where $\alpha_1$ and $\alpha_2$ are constants of integration and $t_0$ can be chosen. Recalling that $$y(t) = A_1(t)y_1(t) + A_2(t)y_2(t)$$ we obtain 
$$ y(t) = \int_{t_0}^{t} \bigg( \frac{y_1(x) y_2(t) - y_1(t) y_2(x)}{W(x)} \bigg) f(x)dx +\alpha_1 y_1(t) + \alpha_2 y_2(t).$$
This formula is called the variation of parameters or variation of constants formula. 
\begin{itemize}
\item $y$ is the general solution of $$y''+a_1y'+a_0y = f$$
\item $a_1$, $a_0$ and $f$ are given functions.
\item $y_1$, $y_2$ are the solutions of the homogeneous equation $$y''_j +a_1y'_j +a_0y_j = 0, \hspace{6pt} j = 1,2$$
\item $W$ is the Wronskian $$ W = \det \begin{bmatrix}
y_1 \hspace{5pt} y_2 \\
y'_1 \hspace{5pt} y'_2\\
\end{bmatrix} = y_1y'_2 - y'_1y_2$$
\end{itemize}
\begin{example}
Find the general solution of the equation $$y'' - y = 1.$$
\begin{solution}
The corresponding homogeneous equation is $y''-y = 0$, with auxiliary quadratic $\lambda^2 - 1 = 0$ having roots $\lambda = \pm 1$. Thus the functions $$\begin{cases}
y_1(x) = \exp(+1x) \\
y_2(x) = \exp(-1x) \\
\end{cases}$$ 
are two linearly independent solutions of the homogeneous equation.  Hence $$W = \det \begin{bmatrix}
y_1 \hspace{5pt} y_2 \\
y'_1 \hspace{5pt} y'_2\\
\end{bmatrix} = \det \begin{bmatrix}
\exp(x) \hspace{5pt} \exp(-x)\\
\exp(x) \hspace{5pt} -\exp(-x)\\
\end{bmatrix} = -2.$$ For this example $f(x) = 1$. The variation of parameters formula gives $$y(t) = \int_{t}^{t_0} \frac{\exp(x)\exp(-t)-\exp(t)exp(-x)}{(-2)}*1dx + \alpha_1\exp(t)+\alpha_2\exp(-t)$$ and we chose $t_0 = 0$ for convenience. Thus $$ y(t) = \frac{-1}{2}\exp(-t) \int_{0}^{t}exp(x)dx +\frac{1}{2}\exp(t)\int_{0}^{t}\exp(-x)dx +\alpha_1\exp(t) +\alpha_2\exp(-t)$$ which yields $$y(t) = \frac{-1}{2}\exp(-t)[exp(t)-1]+\frac{1}{2}\exp(t)[1-exp(-t)]+\alpha_1\exp(t)+\alpha_2\exp(-t)$$ $$ = -1+(\alpha_1 + \frac{1}{2})\exp(t) +(\alpha_2 +\frac{1}{2})\exp(-t).$$ Hence $$y(t) = -1 +A_1\exp(t) + A_2\exp(-t).$$ Where $A_1$ and $A_2$ are constants.
\end{solution}
\end{example}

\bigskip

\begin{example}
Find the general solution of $$\frac{d^2y}{dt^2} - 4\frac{dy}{dt} + 4y = t\exp(2t).$$
\begin{solution}
Our auxiliary quadratic is $$\lambda^2 -4\lambda + 4 = 0 \mbox{ or } (\lambda - 2)^2 = 0.$$ One solution of the homogeneous equation is $y_1(t) = \exp(2t)$ and the second is $y_2(t) = t\exp(2t).$ Thus $$ W = \begin{bmatrix}
\exp(2t) \hspace{44pt} t\exp(2t) \\
2\exp(2t) \hspace{12pt} (1+2t)\exp(2t)\\
\end{bmatrix} = \exp(4t)$$ Also $f(t) = t\exp(2t)$. The variation of parameters formula gives $$y(t) = \int_{0}^{t}\frac{\exp(2X)t\exp(2t)-\exp(2x)x\exp(2t)}{\exp(4(x)}x\exp(2x)dx + (\alpha_1+\alpha_2t)\exp(2t)$$ $$ =\int_{0}^{t}(tx\exp(2t) - x^2\exp(2t))dx + (\alpha_1+\alpha_2t)\exp(2t)$$ $$= t\exp(2t)\bigg[\frac{x^2}{2}\bigg]_{0}^{t} -\bigg[\frac{x^3}{3}\bigg]_{0}^{t}\exp(2t) +(\alpha_1+\alpha_2t)\exp(2t).$$ Finally $$y(t) = \frac{1}{6}t^3\exp(2t)+(\alpha_1+\alpha_2t)\exp(2t).$$
\end{solution}
\end{example}

\begin{theorem}
Let $y_1$ and $y_2$ be linearly independent solutions of $$\frac{d^2 y}{dt^2}+ a_1\frac{dy}{dt}+a_0y = 0 $$ and let $y_p$ be any solution of the equation $$\frac{d^2 y}{dt^2}+ a_1\frac{dy}{dt}+a_0y = f, $$
then every solution $y$ of this equation has the form $$y(t) =y_p (t) + A_1y_1(t) + A_2y_2(t)$$ for the appropriate constants $A_1$ and $A_2$, furthermore for every pair of constants $A_1$ $A_2$ the function $y$ defined above and solves the homogeneous equation.The Proof for this will be given in year 2.
\end{theorem}
\subsubsection*{Guidelines for Guessing $\mathbf{y_p}$}
\begin{enumerate}
\item If $f(t) = \alpha \exp (\beta t)$, where $\beta$ is not a root of the auxiliary quadratic, then there exists a particular solution $y_p$ of the form $$y_p(t) = \gamma \exp (\beta t).$$ This is proven below.
$$\frac{d^2y_p}{dt^t}=\beta^2 y_p$$ and so$$(\beta^2 + a_1\beta +a_0)y_p = \alpha \exp (\beta t)$$ by canceling the common factor $\exp (\beta t)$ we get $$(\beta^2 + a_1\beta +a_0)\gamma = \alpha \neq 0$$ since $\beta$ is not a root of the auxiliary quadratic. Hence,
$$\gamma = \alpha (\beta^2 + a_1\beta +a_0).$$
\item Suppose that the auxiliary quadratic has distinct root $\lambda_1 \neq \lambda_2$ and that $$f(t) = \alpha \exp (\lambda_1 t)$$ then there exists a solution of the form $$y_p(t) \gamma t \exp (\lambda_1 t).$$ This again is proven below. $$\frac{dy_p}{dt}=\gamma \exp (\lambda_1 t) [1 + \lambda_1 t]$$
$$\frac{d^2y_p}{dt^2}=\gamma \exp (\lambda_1 t) [2\lambda_1 + \lambda_1^2 t]$$ Hence, $$\frac{d^2 y_p}{dt^2}+ a_1\frac{dy_p}{dt}+a_0y_p $$ $$=\gamma \exp (\lambda_1 t) [2\lambda_1 + \lambda_1^2 t + a_1(\lambda_1 t +1) + a_0 t]$$ $$= \gamma \exp (\lambda_1 t) [t (\lambda_1^2 + a_1\lambda_1 + a_0) + 2\lambda+1 +a_1]$$
we want this to be equal to $$f(t) = \alpha \exp (\lambda_1 t)$$ which as $$(\lambda_1^2 + a_1\lambda_1 + a_0) = 0 $$ this holds precisely when $$\gamma(2 \lambda_1 + a_1)= \alpha $$ as $a_1 = -\lambda_1 -\lambda_2$ and as a result $$\gamma (\lambda_1-\lambda_2) = \alpha$$ so
$$\gamma = \frac{\alpha}{(\lambda_1-\lambda_2}.$$
\item This deals with the case when the auxiliary quadratic has roots $\lambda_1 = \lambda_2$ and $$f(t)= \alpha t + \mu) \exp (\lambda_1 t)$$ this particular solution has the form  $$y_p(t) = q(t) \exp (\lambda_1 t) $$ of the form $$q (t) = a t^3  + bt^2$$
\item If $$f(t) = \alpha_1 \sin (\beta_1 t) + \alpha_2 \cos (\beta_2 t)$$ where $\beta_1$ and $\beta_2$ are real numbers and $i\beta_1$ and $i\beta_2$ are not solutions of the auxiliary quadratic. Then there exists a solution $$y_p(t) = [a_1 \sin (\beta_1 t) + c_1 \cos (\beta_1 t ) + a_2 \sin (\beta_2 t) + c_2 \cos (\beta_2 t ) ]$$ where $a_1, c_1, a_2, c_2$ are to be found.
\end{enumerate}
\section{Coupled Linear Systems}
We are interested in equations of the form $$\frac{dx}{dt} = ax+by$$ $$\frac{dy}{dt} = cx+dy$$ in which $a$, $b$, $c$ and $d$ are constant.
\subsubsection*{Solution Method 1}
Reduce to a single, second order equation. Differentiating the second equation you get $$\frac{d^2y}{dt^2}= c\frac{dx}{dt}+d\frac{dy}{dt} = c(ax+by)+d\frac{dy}{dt}$$ $$ = acx+bcy +d\frac{dy}{dt}.$$ From our second equation, if $c\neq 0$ then $$x = \frac{1}{c}\frac{dy}{dt}-\frac{d}{c}y$$ and so $$\frac{d^2y}{dt^2}= a\frac{dy}{dt}-ady+bcy+d\frac{dy}{dt}$$ or equivalently 
$$\frac{d^2y}{dt^2}-(a+d)\frac{dy}{dt}+(ad-bc)y=0.$$
If we write this system as $$\begin{bmatrix}
\frac{dx}{dt}\\
\vspace{10pt}
\frac{dy}{dt}\\
\end{bmatrix} = \begin{bmatrix}
a \hspace{10pt} b \\
c \hspace{10pt} d \\
\end{bmatrix} * \begin{bmatrix}
x \\
y \\
\end{bmatrix} =:A$$ Then $$\frac{d^2y}{dt^2}-trace(A)\frac{dy}{dt}+\det(A)y = 0.$$
If $c \neq 0$ then $x$ is obtained from $$x = \frac{1}{c}\frac{dy}{dt}-\frac{d}{c}y.$$ If $c = 0$ and $b \neq 0$ then elimiate $y$ and obtain $$\frac{d^2x}{dt^2}-trace(A)\frac{dx}{dt}+\det(A)x = 0.$$ Then if $c=0=b$ and $$\frac{dx}{dt} = ax$$ $$\frac{dy}{dt} =dy$$ therefore they are no longer coupled.
\subsubsection*{Solution Method 2}
Let $z = \begin{bmatrix}
x \\ 
y
\end{bmatrix}$ so that $$\frac{dz}{dt}= Az, \hspace{15pt} A = \begin{bmatrix}
a \hspace{10pt} b \\
c \hspace{10pt} d \\
\end{bmatrix}$$ We look for a solution $z(t) = v \exp(\lambda t)$ where $v$ is a constant vector and $\lambda \in \mathbb{C}.$ Then $$ \frac{dz}{dt} = v \lambda\exp(\lambda t) \implies Az = Av \exp(\lambda t)$$ which is equivalent to $$Av = \lambda v.$$
This means that $v$ must be an eigenvector of $A$ with eigenvalue $\lambda.$ The eigenvalues of a matrix $A$ satisfy $$\det(A - \lambda I ) = 0 $$ or  $$\lambda^2 -(a+d)\lambda +(ad-bc) = 0 $$ this is the same as the auxiliary quadratic of second order equation arising from method one.

\begin{example}
\smallskip
Solve the coupled linear equations;
$$\frac{dz}{dt} = Az, \hspace{15pt} A=\begin{bmatrix}
-2 &1 \\
1 & -2 \\
\end{bmatrix}$$
\begin{solution}
The eigenvalues of $A$ solve, $$\lambda^2 +4\lambda +3 = 0 $$ this factorises to give $\lambda = -3$ and $-1$ then we need the eigenvector for each, first $\lambda = -1$ this eigenvector satisfies $$\begin{bmatrix}
-2- (-1) & 1\\
1 & 2-(-1)\\
\end{bmatrix}v =0$$ so we take  $$v = \alpha \begin{bmatrix}
\frac{1}{\sqrt[]{2}}\\
\frac{1}{\sqrt[]{2}}\\
\end{bmatrix} \hspace{15pt} \alpha \in \mathbb{C}.$$
Some true for $\lambda -3$ that gives 
$$v = \alpha \begin{bmatrix}
\frac{1}{\sqrt[]{2}}\\
\frac{1}{\sqrt[]{2}}\\
\end{bmatrix} \hspace{15pt} \alpha \in \mathbb{C}$$
\end{solution}
\end{example}


%------------------------------------------------
\endinput
