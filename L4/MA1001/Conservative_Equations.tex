% !TEX root = main.tex

%------------------------------------------------
\chapter{Conservative Equations}
\begin{definition}
A conservative equation is an equation of the form $$\frac{d^2x}{dt^2} +V'(x) = 0.$$
\end{definition}
\begin{remark}
This means that we can consider any equation $$\frac{d^2x}{dt^2} + F(x) = 0$$ in which we can find an indefinite integral of $F.$
\end{remark}
\subsection*{Trick for Conservative Equation}
Multiply the equation by $\frac{dx}{dt}$ to obtain $$\frac{dx}{dt}\frac{d^2x}{dt^2}+V'(x)\frac{dx}{dt} = 0$$ or by the chain rule, $$\frac{d}{dt}\bigg[\frac{1}{2}\bigg(\frac{dx}{dt}\bigg)^2+V(x)\bigg] = 0.$$ Hence $$\frac{1}{2}\bigg(\frac{dx}{dt}\bigg)^2+V(x) = E$$ where $E$ is a constant. We could rearrange this as $$\frac{dx}{dt} = \pm \sqrt[]{2E-2V(x)}.$$ This is an infinite family  (parametrized by E) of separable first order equations. Unfortunately this usually does not help.
\begin{example}
Consider a pendulum of length $l$ with all the concentrated at the end. The equation governing the motion of the pendulum is $$\frac{d^2\theta}{dt^2} +\frac{g}{l}\sin(\theta) = 0$$ where $g$ is the acceleration due to gravity. Prove that there exisits $T>0$ such that $\theta (t+T) = \theta(t)$ for all $t.$ The least such $T$ is called the period of $\theta.$ So $$V'(\theta) = \frac{g}{l}\sin(\theta)$$ and without loss of generality we take $$V(\theta) = \frac{-g}{l}\cos(\theta)$$ and the energy conservation equation becomes $$\frac{1}{2}\bigg(\frac{dx}{dt}\bigg)^2- \frac{g}{l}\cos(\theta) = E.$$ In the special case where $E = \frac{g}{l}$ we get $$\frac{1}{2}\bigg(\frac{dx}{dt}\bigg)^2 = \frac{g}{l}(1+\cos(\theta) = \frac{2g}{l}\cos^2\bigg(\frac{\theta}{2}\bigg).$$ However this is the only cases where an explicit solution is possible. For the other cases we shall plot the curves in $\mathbb{R}^2$ given by $t \mapsto(\theta(t), \frac{d\theta}{dt}).$
\end{example}
\begin{definition}
Consider a conservative equation $$\frac{d^2x}{dt^2} + V'(x) = 0.$$ The phase diagram for this equation is the set of parametric curves $$t \rightarrow(x(t),x'(t))$$ where $x$ solves the differential equation.
\end{definition}
\begin{example} [Linearized Pendulum]
From the previous example $$\frac{d^2x}{dt^2}+\frac{g}{l}x = 0$$ has an auxiliary quadratic $$\lambda^2 + \frac{g}{l} = 0$$ with roots $$\lambda = \pm i \omega, \hspace{15pt} \omega = \sqrt[]{\frac{g}{l}}.$$ The general solution is $x(t) = A\sin(\omega t) + B\cos(\omega t)$ where $A$, $B$ are constants. We can also write this as $x(t) = R\cos(\omega t - \delta).$ Then $$R\cos(\omega t - \delta) = R\cos(\omega t )\cos(\delta)+R\sin(\omega t)sin(\delta)$$ and thus $$R\cos(\delta) = B, \hspace{15pt} R\sin(\delta) = A.$$ Hence $$x'(t) = -\omega R \sin(\omega t -\delta)$$ and so $$(x(t),x'(t) = (R\cos(\omega t - \delta), -\omega R\sin(\omega t -delta)).$$ These are ellipses because $$\bigg(\frac{x(t)}{R}\bigg)^2+\bigg(\frac{x'(t)}{\omega R}\bigg)^2 = 1$$ The fact that the solutions are periodic is revealed by the fact that the curves in the phase diagram are closed.
\end{example}
\begin{example}[Pendulum Phase Diagram]
For the lineraized pendulum $$\frac{d^2x}{dt^2}+\frac{g}{l}x=0.$$ We have $$V'(x)=\frac{g}{l}x = \omega^2x.$$ We can take $$V(x) = \frac{1}{2}\omega^2x^2$$ and the energy conservation equation is $$\frac{1}{2}(x'(t))^2+V(x) = E$$ where $E$ is a constant i.e $$\frac{1}{2}\omega ^"(x(t))^2+\frac{1}{2}(x'(t))^2 = E.$$ Then by multiplying with 2 and dividing by $2E$ to get $$\bigg(\frac{x(t)\omega}{\sqrt[]{2E}}\bigg)^2+\bigg(\frac{x'(t)}{\sqrt[]{2E}}\bigg)^2 = 1$$ which is the same as the previous example with $$R =\frac{\sqrt[]{2E}}{\omega}.$$ For the full pendulum equation $$V'(x) = \frac{g}{l}\sin(x) = \omega^2\sin(x)$$ and we take $$V(x) = \omega^2 - \omega^2 \cos(x).$$ The phase curves are $$\frac{1}{2}(x'(t))^2+\omega^2(1-\cos(x))=E.$$
 We want to plot the curves, and we already have the case $$V(x) = \frac{1}{2}x^2$$ or generally $$V(x) = kx^2 \hspace{25pt} k>0$$ when the phase curves are ellipses. In both of these examples, $V(x)  \geq 0$ everywhere so we need $E\leq 0$. When $E=0$ we need both $y=0$ and $v(x) = 0$. For the case $V(x) = \frac{g}{l}(1-\cos(x))$ this gives the points $(2n\pi,0), n\in \mathbb{Z}.$ For the case $V(x) = kx^2$ we just get (0,0) when $E>0$ is small. Thus we draw a line at height $E$ across the graph $V(x).$ 
 \smallskip
 For the pendulum equation when $E\geq \frac{2g}{l}$ then $E\geq V(x)$ for all $x$. 
 
 \bigskip 
 
 Suppose we now wish to approximate the phase curves, either for the pendulum equation or any other, in a small neighborhood of a point $(x_0,0)$ at which $V$ has a local maximum e.g $x_0 = \pi$ for the pendulum equation with $V(x) = \frac{g}{l}(1-\cos(x))$. We use the Taylor expansion $$V(x) = V(x_0) +(x-x_0)+V'(x_0)+\frac{1}{2!}(x-x_0)^2V''(x_0)+....$$ Since $V$ has a local maximum at $x_0$ $$V'(x_0) = 0$$ $$V''(x_0) \leq 0.$$ Assume $V''(x_0) <0$ and so the energy conservation equation is $$E=\frac{1}{2}(x')^2+V(x) = \frac{1}{2}(x')^2+V(x_0) + \frac{1}{2|}(x-x_0)^2V''(x_0) +.....$$ $$\approx \frac{1}{2}(x')^2-\frac{1}{2}\omega^2(x-x_0)^2$$ where $\omega^2 = -V''(x_0)>0.$ The curves $\frac{1}{2}y^2-\frac{1}{2}\omega^2(x-x_0)^2 = E$ are hyperbola.
\end{example}
\begin{example}
Draw the phase diagram for the equation $$\frac{d^2x}{dt^2}+2x-3x^2 = 0$$ show that there exists periodic solutions and obtain an integral expression for their periods.
\begin{solution}
Here $V'(x) = 2x-3x^2$ so $V(x) = x^2-x^3$ is a suitable choice for $v$. The energy equation is $$\frac{1}{2}(x')^2+x^2+x^3 = E$$ 
\end{solution}
\end{example}
\subsection*{Integral Expression for the Period of a Periodic Solution}
Observe that since $\frac{1}{2}(x')^2+V(x)$ is a constant and since $x'=0$ when $x=\alpha$ and $x=\beta$ we have $$V(\alpha)=v(\beta).$$ The period of the solution is the time it takes from $\alpha$ to $\beta$ plus the time to return. By symmetry these times are equal so the period is twice the transit from $\alpha$ to $\beta.$ Thus the period is $$T= \int_0^T dt = 2\int_\alpha^\beta\frac{dt}{dx}dx = 2\int_\alpha^\beta \frac{dx}{x'}$$ Now $$\frac{1}{2}(x')^2 +V(x) = V(\alpha)=v(\beta)$$ giving $$x' = \pm \sqrt[]{2(V(\beta)-V(x))}.$$ From $\alpha$ to $\beta$, $x$ increases so $$x' = \sqrt[]{2(V(\beta)-V(x))}.$$ Thus $$T = \sqrt[]{2}\int_\alpha^\beta\frac{dx}{\sqrt[]{V(\beta)-V(x)}}.$$


%------------------------------------------------
\endinput
